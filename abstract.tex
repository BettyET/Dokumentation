\cleardoublepage
\chapter*{Abstract}


	NTP-synchronized clocks have the ability to send a message through the network. It seems useful to
	send an alarm message if a clock has been damaged through vandalism. The aim of this paper is to find out if
	it's possible to detect vandalism with means of acceleration measurement. 
	
	To do that, an
	experimental setup was developed. The central components were three piezoelectric acceleration sensors. The
	experiments with different projectiles provided information about the optimal sensor position, the 
	dimesion an shape of the acceleration signals and if there is a chance to distinguish vandalism from a passing
	train. As a next step a MEMS acceleration sensor has been taken in service as a data logger. This
	showed that the shock detector can be easily and cheaply implemented. Different ideas to integrate the
	shock detector into the existing clock set have been collected. After discussing the pros and cons
	of each idea, the best one has been realized as a shock detector module. 
	
	It's a likely assumption
	that clocks get hit through an object like a stone. Even weak strikes result in large
	accelerations measured by the clockwork. The strike deforms the plastic disk. The air between the
	disk and the clock face is compressed an acts like a spring. Thats the reason why the clock face begins to 
	oscillate at a specific frequency. In comparison to that, the force initiated from a passing by
	train or gust of wind, has an effect on the entire surface of the clock face. The plastic disk doesn't
	deform and the clock face doesn't get stimulated. As a consequence the optimal sensor position is the
	clockwork, which is connected to the clock face. The MEMS acceleration sensor H3LIS331DL with a
	package size of just 3x3x1mm$^3$ is used for realization.
		

%	NTP-synchronisierte Aussenuhren bieten die M�glichkeit, eine Nachricht �ber das Netzwerk
%	abzusetzen. Es ist sinnvoll, dass bei einer durch Vandalismus besch�digten Aussenuhr ein
%	Alarm-Signal �ber das Netzwerk abgegeben werden kann. Es ist abzukl�ren, ob die Erkennung eines
%	solchen Shock Events mittels Beschleunigungsmessung grunds�tzlich m�glich ist.
%	
%	Es wurde ein Messaufbau mit drei piezoelektrischen Beschleunigungssensoren entwickelt, um Shock
%	Events zu simulieren. In Versuchen wurde die optimale Position des Beschleunigungssensors in der
%	Aussenuhr gesucht und abgekl�rt, ob die Detektion eines Shock Events zuverl�ssig realisiert werden
%	kann und beispielsweise von einer Zugdurchfahrt unterscheidbar ist. Zus�tzlich wurde ein MEMS
%	Beschleunigungssensor in Betrieb genommen und ebenfalls am Versuchsaufbau getestet. Verschiedene
%	Ideen, wie ein Shock Detector in den bestehenden Uhrenaufbau integriert werden kann, wurden
%	skizziert und die Vor- und Nachteile diskutiert. Zum Schluss wurde eine Realisierungsidee
%	umgesetzt.
%	
%	Es ist wahrscheinlich, dass bei Vandalismus Aussenuhren mit einem Gegenstand beworfen oder
%	geschlagen werden. Die Versuche haben gezeigt, dass bereits bei schwachen Schl�gen hohe
%	Beschleunigungen an der Aussenuhr zu messen sind. Durch den Schlag wird die sch�tzende
%	Scheibe eingedr�ckt und das dahinter liegende Zifferblatt angeregt, welches nun mit einer
%	bestimmten Frequenz schwingt. Im Gegensatz dazu wirkt die Kraft- ausgel�st durch eine Zugdurchfahrt
%	oder Windb�e- auf der ganzen Scheibenoberfl�che. Die Scheibe wird dadurch kaum deformiert und das
%	Zifferblatt somit nicht angeregt. Die geeignetste Position f�r einen Beschleunigungssensor ist am
%	Uhrwerk. F�r die Realisierung des Shock Detectors wird ein MEMS Beschleunigungssensor
%	eingesetzt. Dieser hat eine Gr�sse von nur gerade 3x3x1mm3.
