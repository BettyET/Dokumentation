\cleardoublepage \preface

Sensoren werden in unserem Alltag und in der Industrie immer wichtiger. Eine Verbindung zwischen Umwelt und Maschine 
wird so erschaffen. Dies erlaubt der Maschine, auf Ver�nderungen in der Umwelt zu reagieren. Technologien wie die MEMS-Technologie
erm�glichen kleine und g�nstige Sensorl�sungen. Beschleunigungssensoren bieten unz�hlige Anwendungsm�glichkeiten. 
Sie werden in Smartphones, Kameras, Spielzeugen, aber auch in Airbag Systemen oder zum Messen von Vibrationen oder Schocks 
bei Maschinen verwendet. Auch bei NTP-synchronisierten Aussenuhren soll nun die M�glichkeit gepr�ft werden, Vandalismus mittels Beschleunigungsmessung
zu detektieren. Gelingt dies, k�nnen Aussenuhren preisg�nstig mit dem neuartigen Feature "'Schock Detektion"' erweitert werden. 
 

\italictitle{Danksagung}
Ich bedanke mich bei der Moser-Baer AG f�r die angenehme Zusammenarbeit und hierbei besonders bei den 
beiden Herren Michael Sommer und Hansj�rg Rohrer. Sie waren meine Ansprechpartner bei der Moser-Baer AG und
bei meinen Fragen stets sehr hilfsbereit. Weiter m�chte ich mich bei Herrn Prof. Zeno St�ssel f�r die 
fachkompetente Betreuung seitens der Hochschule Luzern Technik \& Architektur bedanken. 
Zu Dank verpflichtet bin ich auch Tobias Pl�ss f�r die Latex-Vorlage und Prof. Leo Suter f�r 
das Lektorat. 

\medskip
\medskip
\doclocation, den \today

\medskip
die Autorin \\
\docauthor


