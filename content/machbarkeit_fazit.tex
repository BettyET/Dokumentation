\chapter{Fazit der Machbarkeitsstudie}
	Ein Schlag auf die Scheibe der Aussenuhr hat schnell grosse Beschleunigungen im dreistelligen
	Bereich zur Folge. Die beste Position f�r den Beschleunigungssensor ist am oder im Uhrwerk.
	Unabh�ngig vom der Position des Schlags auf der Scheibe werden dort die gr�ssten Beschleunigungen
	gemessen. Der Schlag auf die Scheibe hat neben einer Verbiegung der ganzen Uhr eine Einbuchtung der
	Scheibe und damit eine Komprimierung der Luft zwischen Scheibe und Ziffernblatt zur Folge. Deshalb
	wird das Ziffernblatt angeregt, an welchem das Uhrwerk und somit der Beschleunigungssensor
	befestigt ist. Im Gegensatz dazu wird durch den Aufprall eines Softballs fast
	ausschliesslich die Uhr verbogen. Dies ist auf die Gr�sse der Kontaktfl�che bei der Kraft�bertragung zwischen
	Ball und Scheibe zur�ck zu schliessen.
	Der am Uhrwerk platzierte Sensor erf�hrt durch die fehlende Einbuchtung der Scheibe eine um
	Faktoren geringere Beschleunigung. Da die Sog-Druckwelle des Zuges auf der ganzen Fl�che
	angreift, wird erwartet, dass keine Einbuchtung der Scheibe zustande kommt. Zus�tzlich ist die
	Dauer der Anregung durch den vorbeifahrenden Zug zu lang, um die Uhr in Schwingung zu versetzen.
	
	Die Inbetriebnahme des Datenloggers mit dem MEMS-Sensors H3LIS331DL von STMicroelectronics hat gezeigt, dass 
	Schock Detektion mit einem g�nstigen Sensor und einfachem Algorithmus realisierbar ist. Ein Schwellwert gen�gt, 
	um den Schock zu detektieren. Abschliessende 
	Tests der Schock Detektion an der Aussenuhr haben die Machbarkeit auch im Versuch best�tigt. 

