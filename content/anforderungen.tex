\chapter{Anforderungen}

Die folgende Liste zeigt die zu Beginn der Arbeit mit dem Industriepartner definierten
Anforderungen. Diese sind in Anforderungen an die Machbarkeitsstudie, und die Anforderungen an das
Produkt unterteilt. Die Anforderungen an das Produkt sind f�r die Machbarkeitsstudie noch nicht
relevant.
\\
\begin{longtable}{p{0.5cm} p{0.5cm} p{8.5cm} p{2cm} } \toprule
	\textbf{Nr.}	&  &\textbf{Anforderung} & \textbf{Priorit�t} \\	
	\midrule
	\endhead
	\multicolumn{3}{l}{\emph{Fortsetzung auf n�chster Seite}}	\\ \bottomrule \endfoot \endlastfoot
	\\
	  &		&\multicolumn{2}{l}{\textbf{Anforderungen an die Machbarkeitsstudie}}							
	\\
	1 & 	&	Es ist zu �berpr�fen, ob mit einem Beschleunigungssensor 
				Vandalismus an einer Uhr (Besch�digung Scheibe, Uhrwerk) 
				erkannt werden kann												& Forderung \\ 
	2 &  	& 	Der Vandalismus muss von Windb�en und der Sog-Druckwelle 
				eines Zuges unterschieden werden k�nnen. 						& Forderung \\		
	3 & 	& 	Es sollen verschiedene Schock-Situationen getestet werden.		& Forderung \\
	  & 3.1	&	Es soll ein geeigneter Messaufbau entwickelt werden, um 
				die Schock-Situationen zu simulieren. 							& Forderung \\
	  & 3.2 & 	Der Messaufbau muss so gew�hlt werden, dass die Messungen 
				reproduzierbar sind. 											& Forderung \\
	4 & 	&	Falls die Detektion m�glich ist, soll ein geeigneter 
				Algorithmus entwickelt werden (z.B. in MATLAB). 				& Forderung \\
	5 & 	& 	Es soll abgekl�rt werden, ob die Implementierung auf dem
				vorhandenem ATMEL- Kontroller realisiert werden kann.			& Forderung \\
	  &	5.1	&	Falls die Implementierung auf dem ATMEL- Kontroller nicht
				realisiert werden kann, muss abgekl�rt werden, welche
				anderen Komponenten ben�tigt werden.							& Forderung \\
	6 &		& 	Der Stromverbrauch der Produkterweiterung soll abgekl�rt und
				dokumentiert werden.											& Forderung \\
	7 & 	& 	Es soll abgekl�rt werden, ob eine vorhandene Schnittstelle 
				(SPI, I2C, UART) verwendet werden kann.							& Forderung \\
	\\
	 &		&\multicolumn{2}{l}{\textbf{Anforderungen an das Produkt}}								
	\\
	1 &		& 	Die Sensorik darf von aussen nicht sichtbar sein.				& Forderung \\
	2 & 	& 	Der Alarm soll mittels SNMP �ber die bestehende Uhrsteuerung
				abgesetzt werden k�nnen.										& Forderung \\
	3 &		&	Die Kosten f�r den Sensor liegen im Breich von 5 bis maximal 10 Fr.
				pro St�ck.														& Forderung \\
	4 & 	& 	Das Sensormodul soll einfach in der Uhr montiert werden k�nnen.
				Die Platzverh�ltnisse sind insofern beschr�nkt, dass die
				Beleuchtung der Uhr nicht gest�rt wird (siehe Anforderung
				Produkt Nr. 1)													& Forderung \\
	5 & 	&	Das Produkt muss bei einer Betriebstemperatur von -30 bis +70�C 
				arbeiten.  														& Forderung \\
	\bottomrule					 	
	\caption{Anforderungen an die Machbarkeitsstudie und an das Produkt} 
	\label{tab:Anforderungen}
\end{longtable}