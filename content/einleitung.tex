\chapter{Einleitung}
	\section{Aufgabenstellung}
	NTP-synchronisierte Aussenuhren bieten die M�glichkeit, eine Nachricht �ber das Netzwerk
	abzusetzen. Es ist sinnvoll, dass bei einer durch Vandalismus besch�digten Aussenuhr ein
	Alarm-Signal �ber das Netzwerk abgegeben werden kann. Es ist abzukl�ren, ob die Erkennung eines
	solchen Schock Events mittels Beschleunigungsmessung grunds�tzlich m�glich ist. Ein geeigneter
	Beschleunigungssensor soll daf�r evaluiert werden. Ein Messaufbau muss entwickelt werden, um
	Schock-Signale zu simulieren. Die Messungen mit diesem Versuchsaufbau sollen m�glichst
	reproduzierbar sein. Es muss abgekl�rt werden, ob die Detektion zuverl�ssig ist und Vandalismus von
	Windb�en und der Sogdruckwelle eines Zuges unterschieden werden k�nnen. Ein geeigneter Algorithmus
	f�r die Detektion soll entwickelt werden. Es sollen Abkl�rungen gemacht werden, wie ein Schock
	Detektor am besten in den bestehenden Uhrenaufbau integriert werden kann.
	
	Die urspr�ngliche schriftliche Aufgabenstellung befindet sich im Anhang \ref{chp:aufgabenstellung}. Die oben beschriebene
	Aufgabenstellung entspricht den definierten Zielen aus der ersten Besprechung zu Beginn der Arbeit mit der
	Firma Moser-Baer AG. Eine ausf�hrliche Anforderungsliste befindet sich im Schlussteil \ref{chap:anforderungen}.
	
	\section{Inhalt der Arbeit}
	Der erste Teil der Arbeit beinhaltet die Grundlagen der Arbeit. Die n�tigen physikalischen Begriffe
	werden erl�utert, bekanntes zu Sog-Druckwellen von Z�gen vorgestellt. Es wird eine �bersicht zu den
	verschiedenen Beschleunigungssensoren mit ihren Eigenschaften gegeben.
	Der Hauptteil kl�rt einerseits die Machbarkeit und beantwortet somit die Frage, ob die Detektion von 
	Schock Events mittels Beschleunigungssensoren m�glich ist. Andererseits wird die Umsetzung eines
	ersten Schock Detektor Moduls dokumentiert. In einem letzten Teil werden die gewonnen Erkenntnisse 
	zusammengefasst und diskutiert. 
	