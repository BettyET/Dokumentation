%-------------------------------------------------------------------------------
% $HeadURL: http://hb9etc.ch/svn/pluess/tex/da_doku/anhang_glossar.tex $
% $Revision: 861 $
% $Author: tobias $
% $Date: 2013-12-23 21:15:48 +0100 (Mon, 23 Dec 2013) $
%-------------------------------------------------------------------------------

\chapter{Glossar} \index{Glossar} \index{Abk�rzungsverzeichnis}
%\markboth{Glossar}{Glossar}
%\addcontentsline{toc}{chapter}{Glossar}
\label{glossar}

\begin{description}[style=multiline,leftmargin=2.5cm]
  \item[\bf Queue] Eine Liste mit Eintr�gen, welche abgearbeitet werden sollen. Eine Queue kann
  einen Aufrufer blockieren, falls sie voll oder leer ist. \index{Queue} \index{Queue|see{Queue}}
  
  \item[\bf Mounten] engl. \emph{mount "'montieren"', "'befestigen"'}. Bezeichnet bei Unix sowie
  einigen anderen Betriebssystemen den Vorgang, ein Dateisystem an einer bestimmten Stelle - dem
  Mountpoint - verf�gbar zu machen, damit der Benutzer auf die Dateien zugreifen kann.
  \index{Mounten}
  
  \item[\bf pr�emptiv] engl. \emph{preemptive "'unterbrechend"'}. Der Task mit der h�chsten
  Priorit�t erh�lt die CPU. Tasks gleicher Priorit�t teilen sich die Rechenzeit. Der Scheduler kann 
  einen Task unterbrechen. \index{pr�emptiv}
  
  \item[\bf I2C] engl. \emph{Inter-Integrated Circuit}. I2C ist ein 1982 von Philips Semiconductors (heute
  NXP Semiconductors) entwickelter serieller Datenbus.
 
\end{description}
