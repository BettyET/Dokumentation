%-------------------------------------------------------------------------------
%
% Eine \LaTeX-Vorlage zum Buch-Style des Springer-Verlages (svmono.cls)
% f�r die deutsche Sprache
%
% Idealerweise zu verwenden mit pdfLatex / Miktex unter Windows
% zusammen mit TexnicCenter
%
% Anm.: Das Template entspricht zu 95 % dem Template f�r Diplomarbeiten gem.
% http://www.formbuch.de , eine ausf�hrliche Anleitung zum Gebrauch findet sich
% im Buch "Form der wissenschaftlichen Ausarbeitung"  (Springer-Verlag)
%
% Autor:            Tilo Gockel
% Versionsnummer:   0.901
% Datum:            12. Mai 2010
%
% Kontakt:          Hochschule Aschaffenburg, Forschungsreferat
%
%                   info@formbuch.de
%                   http://www.formbuch.de
%
%-------------------------------------------------------------------------------

%\documentclass[envcountsame, envcountchap, openany, deutsch, vecarrow]{svmono}
\documentclass[envcountsame, envcountchap, deutsch, vecarrow,]{svmonosf}
%-------------------------------------------------------------------------------

% fonts and the like
\usepackage{lmodern}
\usepackage{avant}
\usepackage{fouriernc}

\usepackage{svmonoext}
\usepackage{pgfgantt}
\usepackage{pdflscape}
\usepackage{longtable}
\usepackage{lscape}

\usepackage{setspace} 
\usepackage{caption}	% Beschriftungen in Breite einschr�nken
\usepackage{wrapfig}
\usepackage{bigdelim}
\usepackage{multirow}
\usepackage{amsmath}
\usepackage[rightcaption]{sidecap}	%Beschriftungen neben Bildern
\usepackage[]{mcode}				% Einbindung von Matlab-Code

%-------------------------------------------------------------------------------
\usetikzlibrary{calc,patterns,decorations.pathmorphing,decorations.markings}

%-------------------------------------------------------------------------------

\def \doctitle {Schock Detektion f�r vernetzte Uhren }
\def \docauthor {Bettina Wyss}
\def \docsubject {Subjekt}
\def \docsubtitle {Pipapo}
\def \doclocation {Luzern}
\def \dockeywords {{Stichworte} {Stichworte} {Stichworte}}

\def \doccomment {Hier kann eine Widmung oder ein Spruch eingef�gt werden.}

%-------------------------------------------------------------------------------

\hyperrefsetup{colorlinks=false}

\renewcommand{\baselinestretch}{1.2}

\lstset{%
    basicstyle=\ttfamily,%
    keywordstyle=\color{blue},%
    columns=fixed,basewidth=.5em,%
    commentstyle=\color[RGB]{0,127,0},%
    linewidth=\linewidth,%
    captionpos=b,%
    frame = none,%
    backgroundcolor = \color[RGB]{230,230,230},%
    aboveskip=\intextsep,%
    belowskip = \floatsep,%
    showstringspaces = false,%
    stringstyle = \color[RGB]{163,21,21},%
    tabsize=2,%
    float=[htbp]}

%-------------------------------------------------------------------------------

\makeindex

\bibliographystyle{bibliografie/ka-style}

\begin{document}
\onehalfspacing
%\nocite{*}                     % vollst�ndige bibliographie anzeigen

%-------------------------------------------------------------------------------

\begin{titlepage}

\begin{center}

% Oberer Teil der Titelseite:
%\includegraphics[width=0.15\textwidth]{./logo}\\[1cm]    
\textsc{\LARGE Diplomarbeit}\\[0.5cm]
\textsc{\Large BAA+E.F1601 }\\[2.5cm]

\textsc{\Large Hochschule Luzern\\
    ~\\
    Technik \& Architektur}\\[0.5cm]

\vfill{}

% Title
\newcommand{\HRule}{\rule{\linewidth}{0.5mm}}
\HRule \\[0.4cm]
{   
        ~\\
        \huge \doctitle\\[0.8cm]}
		\large %\docsubtitle\\
%[0.6cm]
\HRule \\[4cm]
\vfill
% Author and supervisor
\begin{minipage}{0.4\textwidth}
    \begin{flushleft} \large
        \emph{Autorin:}\\
        Bettina \textsc{Wyss}\\
    \end{flushleft}
     \vspace{2.2cm}
     \begin{flushleft} \large
     	\emph{Abteilung:} \\
     	Elektrotechnik
     \end{flushleft}
     \vspace{2mm}
     \begin{flushleft} \large
     	\emph{Klassifikation:} \\
     	R�cksprache
     \end{flushleft}
\end{minipage}
\hfill
\begin{minipage}{0.4\textwidth}
    \begin{flushright} \large
        \emph{Industriepartner:} \\
        Moser-Baer AG\\
        Spitalstrasse 7\\
        3454 Sumiswald
    \end{flushright}
    \vspace{1.2cm}
    \begin{flushright} \large
    	\emph{Dozent:} \\
    	 Prof. Zeno  \textsc{St�ssel}
    \end{flushright}
    \vspace{1mm}
    \begin{flushright} \large
    	\emph{Experte:} \\
    	Thomas  \textsc{Schmidiger}
    \end{flushright}
\end{minipage}

\vfill{}
\vfill{}
\vfill{}

% Unterer Teil der Seite
{\large Horw\\ \today}

\end{center}

\end{titlepage}

%\cleardoublepage
\chapter*{Sperrvermerk}
\markboth{Sperrvermerk}{Sperrvermerk}

Die vorliegende Dokumentation enth�lt vertrauliche Informationen der
X AG. Ver�ffentlichungen oder Vervielf�ltigungen -- auch
auszugsweise -- sind daher ohne schriftliche Genehmigung der
X AG nicht gestattet. Die Arbeit ist nur den Korrektoren sowie
den Mitgliedern des Pr�fungsausschusses zug�nglich zu machen.

\medskip
\medskip
\doclocation, den \today

\medskip
der Autor \\
\docauthor

\cleardoublepage
\chapter*{Redlichkeitserkl�rung}
\markboth{Redlichkeitserkl�rung}{Redlichkeitserkl�rung}

Hiermit erkl�re ich, dass ich die vorliegende Arbeit selbstst�ndig angefertigt und keine
anderen als die angegebenen Hilfsmittel verwendet habe. S�mtliche verwendeten Textausschnitte,
Zitate oder Inhalte anderer Verfasser wurden ausdr�cklich als solche gekennzeichnet.

\medskip
\medskip
\doclocation, den \today

\medskip
der Autor \\
\docauthor

\cleardoublepage
\chapter*{Abstract}


	NTP-synchronized clocks have the ability to send a message through the network. It seems useful to
	send an alarm message if a clock has been damaged through vandalism. The aim of this paper is to find out if
	it's possible to detect vandalism with means of acceleration measurement. 
	
	To do that, an
	experimental setup was developed. The central components were three piezoelectric acceleration sensors. The
	experiments with different projectiles provided information about the optimal sensor position, the 
	dimesion an shape of the acceleration signals and if there is a chance to distinguish vandalism from a passing
	train. As a next step a MEMS acceleration sensor has been taken in service as a data logger. This
	showed that the shock detector can be easily and cheaply implemented. Different ideas to integrate the
	shock detector into the existing clock set have been collected. After discussing the pros and cons
	of each idea, the best one has been realized as a shock detector module. 
	
	It's a likely assumption
	that clocks get hit through an object like a stone. Even weak strikes result in large
	accelerations measured by the clockwork. The strike deforms the plastic disk. The air between the
	disk and the clock face is compressed an acts like a spring. Thats the reason why the clock face begins to 
	oscillate at a specific frequency. In comparison to that, the force initiated from a passing by
	train or gust of wind, has an effect on the entire surface of the clock face. The plastic disk doesn't
	deform and the clock face doesn't get stimulated. As a consequence the optimal sensor position is the
	clockwork, which is connected to the clock face. The MEMS acceleration sensor H3LIS331DL with a
	package size of just 3x3x1mm$^3$ is used for realization.
		

%	NTP-synchronisierte Aussenuhren bieten die M�glichkeit, eine Nachricht �ber das Netzwerk
%	abzusetzen. Es ist sinnvoll, dass bei einer durch Vandalismus besch�digten Aussenuhr ein
%	Alarm-Signal �ber das Netzwerk abgegeben werden kann. Es ist abzukl�ren, ob die Erkennung eines
%	solchen Shock Events mittels Beschleunigungsmessung grunds�tzlich m�glich ist.
%	
%	Es wurde ein Messaufbau mit drei piezoelektrischen Beschleunigungssensoren entwickelt, um Shock
%	Events zu simulieren. In Versuchen wurde die optimale Position des Beschleunigungssensors in der
%	Aussenuhr gesucht und abgekl�rt, ob die Detektion eines Shock Events zuverl�ssig realisiert werden
%	kann und beispielsweise von einer Zugdurchfahrt unterscheidbar ist. Zus�tzlich wurde ein MEMS
%	Beschleunigungssensor in Betrieb genommen und ebenfalls am Versuchsaufbau getestet. Verschiedene
%	Ideen, wie ein Shock Detector in den bestehenden Uhrenaufbau integriert werden kann, wurden
%	skizziert und die Vor- und Nachteile diskutiert. Zum Schluss wurde eine Realisierungsidee
%	umgesetzt.
%	
%	Es ist wahrscheinlich, dass bei Vandalismus Aussenuhren mit einem Gegenstand beworfen oder
%	geschlagen werden. Die Versuche haben gezeigt, dass bereits bei schwachen Schl�gen hohe
%	Beschleunigungen an der Aussenuhr zu messen sind. Durch den Schlag wird die sch�tzende
%	Scheibe eingedr�ckt und das dahinter liegende Zifferblatt angeregt, welches nun mit einer
%	bestimmten Frequenz schwingt. Im Gegensatz dazu wirkt die Kraft- ausgel�st durch eine Zugdurchfahrt
%	oder Windb�e- auf der ganzen Scheibenoberfl�che. Die Scheibe wird dadurch kaum deformiert und das
%	Zifferblatt somit nicht angeregt. Die geeignetste Position f�r einen Beschleunigungssensor ist am
%	Uhrwerk. F�r die Realisierung des Shock Detectors wird ein MEMS Beschleunigungssensor
%	eingesetzt. Dieser hat eine Gr�sse von nur gerade 3x3x1mm3.



\cleardoublepage
\tableofcontents

\cleardoublepage \preface

\lipsum[1]

\italictitle{Danksagung}
\lipsum[3]

\medskip
\medskip
\doclocation, den \today

\medskip
der Autor \\
\docauthor



\chapter{Einleitung}
	\section{Aufgabenstellung}
	NTP-synchronisierte Aussenuhren bieten die M�glichkeit, eine Nachricht �ber das Netzwerk
	abzusetzen. Es ist sinnvoll, dass bei einer durch Vandalismus besch�digten Aussenuhr ein
	Alarm-Signal �ber das Netzwerk abgegeben werden kann. Es ist abzukl�ren, ob die Erkennung eines
	solchen Schock Events mittels Beschleunigungsmessung grunds�tzlich m�glich ist. Ein geeigneter
	Beschleunigungssensor soll daf�r evaluiert werden. Ein Messaufbau muss entwickelt werden, um
	Schock-Signale zu simulieren. Die Messungen mit diesem Versuchsaufbau sollen m�glichst
	reproduzierbar sein. Es muss abgekl�rt werden, ob die Detektion zuverl�ssig ist und Vandalismus von
	Windb�en und der Sogdruckwelle eines Zuges unterschieden werden k�nnen. Ein geeigneter Algorithmus
	f�r die Detektion soll entwickelt werden. Es sollen Abkl�rungen gemacht werden, wie ein Schock
	Detektor am besten in den bestehenden Uhrenaufbau integriert werden kann.
	
	Die urspr�ngliche schriftliche Aufgabenstellung befindet sich im Anhang \ref{chp:aufgabenstellung}. Die oben beschriebene
	Aufgabenstellung entspricht den definierten Zielen aus der ersten Besprechung zu Beginn der Arbeit mit der
	Firma Moser-Baer AG. Eine ausf�hrliche Anforderungsliste befindet sich im Schlussteil \ref{chap:anforderungen}.
	
	\section{Inhalt der Arbeit}
	Der erste Teil der Arbeit beinhaltet die Grundlagen der Arbeit. Die n�tigen physikalischen Begriffe
	werden erl�utert, bekanntes zu Sog-Druckwellen von Z�gen vorgestellt. Es wird eine �bersicht zu den
	verschiedenen Beschleunigungssensoren mit ihren Eigenschaften gegeben.
	Der Hauptteil kl�rt einerseits die Machbarkeit und beantwortet somit die Frage, ob die Detektion von 
	Schock Events mittels Beschleunigungssensoren m�glich ist. Andererseits wird die Umsetzung eines
	ersten Schock Detektor Moduls dokumentiert. In einem letzten Teil werden die gewonnen Erkenntnisse 
	zusammengefasst und diskutiert. 
	
\part{Grundlagen}
%-------------------------------------------------------------------------------
% $HeadURL: http://hb9etc.ch/svn/pluess/tex/da_doku/adaptfilt.tex $
% $Revision: 861 $
% $Author: tobias $
% $Date: 2013-12-23 21:15:48 +0100 (Mon, 23 Dec 2013) $
%-------------------------------------------------------------------------------

\chapter{Physikalischer Hintergrund} 

Blabla \dots

\section{Beschleunigung} 
Dieser Abschnitt basiert auf dem Unterrichtsskript des Physikunterrichts an der Hochschule 
Luzern \cite{Mechanik}. 
 
Der Ortsvektor $\vec{s}$, die Geschwindigkeit $\vec{v}$ und die Beschleunigung $\vec{a}$ sind
Vektoren, welche einen geometrischen Zusammenhang haben. So ist die Geschwindigkeit die Weg- 
Zuwachs-Rate dadurch die Steigung an die Weg-Zeit-Kurve. 
\begin{equation}
\vec{v}=\frac{d\vec{s}}{dt}
\end{equation}
Die Beschleunigung ist die Geschwindigkeits-Zuwachs-Rate und somit die Steigung an die 
Geschwin-digkeit-Zeit-Kurve. 
\begin{equation}
\vec{a}=\frac{d\vec{v}}{dt}=\frac{d^2\vec{s}}{dt}
\end{equation}
Eine weitere Ableitung f�hrt zum Ruck. Die Ableitungen k�nnen beliebig weitergef�hrt werden. 
Umgekehrt kann die Geschwindigkeit oder die zur�ckgelegte Strecke durch Integration der 
Beschleunigung bzw. Geschwindigkeit berechnet werden. 

Bei der Bahnkurve zeigt der Geschwindigkeitsvektor immer tangential zur Bahnkurve. Der
Beschleunigungsvektor zeigt immer nach innen. Bei einer Kreisf�rmigen Kreisbewegung
zeit der Beschleunigungsvektor zur Mitte des Kreises hin. Diese Beschleunigung nennt
sich Zentripetalbeschleunigung und berechnet sich zu 
\begin{equation}
a_{ZP}=\frac{v^2}{r}= \omega^2\cdot r 
\end{equation} 
mit $v$ der momentanen Geschwindigkeit und $r$ dem Radius des Kreises. 
\newpage
\section{Impuls und Kraftstoss}
Die Beschreibung, wie sich K�rper bewegen, wird mit den drei Axiomen von Newton 
erweitert. Dabei wird die Frage beantwortet, warum sich etwas bewegt. Ein 
K�rper �ndert seine Geschwindigkeit (Beschleunigung) unter Einwirkung einer
Kraft. Dies widerspiegelt sich in der Bewegungsgleichung (zweites Axiom)
\begin{equation}
\vec{F}=m\cdot \vec{a}.
\end{equation}
Eine weitere Beschreibung der Bewegung eines K�rpers ist der Impuls. Er ist das 
Produkt von Masse und Geschwindigkeit. 
\begin{equation}
\vec{p}=m\cdot\vec{v} 
\end{equation}
Das erweiterte Newtonsche Gesetz lautet somit
\begin{equation}
\vec{F}=m\cdot \vec{a} = m\frac{d\vec{v}}{dt}=\frac{d\vec{p}}{dt}
\end{equation}
und l�sst nun auch ver�nderbare Massen zu. Der Kraftstoss $\vec{J}$ ist 
die Fl�che unter der Kraft-Zeit-Kurve. Er entspricht der Impuls�bertragung. 
\begin{equation}
\vec{J} = \int_{t_1}^{t_2} \! \vec{F}(t) \, \mathrm{d}t = \vec{p}_2 - \vec{p}_1
\end{equation}
Aus dem Kraftstoss l�sst sich eine durchschnittlich wirkende Kraft berechnen (\autoref{fig:druchschnittskraft})
\begin{equation}
\vec{F}_{average} = \frac{\vec{J}}{\Delta t} = \frac{\vec{p}_2 - \vec{p}_1}{t_2- t_1} . 
\end{equation}
	\begin{figure}
		\centering
		\includegraphics[width=5.5cm]{img/durchschnittskraft.png}
		\caption[Durchschnittliche Kraft $F_{av}$]{Durchschnittliche Kraft $F_{av}$ \cite{Mechanik}}
		\label{fig:druchschnittskraft}
	\end{figure}
Ein Impuls�bertrag kann bei kurzen, heftigen St�ssen oder langen, sanften St�ssen 
gleich gross sein. Die Form des Impulses h�ngt von der Steifigkeit der Materialien ab, welche
aufeinander treffen. Ein steiferes Material erzeugt einen schmaleren Impuls. Die durchschnittliche 
Kraft $F_av$ wird dadurch gr�sser. Das Energiedichtespektrum wird umso breiter, je schmaler der 
Impuls ist. Diese Zusammenh�nge zeigt die \autoref{fig:kraftstoss}.
	\begin{figure}
		\centering
		\captionsetup{width=10cm}
		\includegraphics[width=10cm]{img/kraftstoss.png}
		\caption[Kraftstoss]{Die Form des Impulses h�ngt von der Steifigkeit des Materials ab. 
			Je steifer ein Material, desto schmaler der Impuls. Daraus folgt, dass das Energiedichtespektrum
			breiter wird \cite{Ruhm}.}
		\label{fig:kraftstoss}
	\end{figure}


%\section{Mechanische Schwingungen}
%
%\section{Elektromechanische Analogie}

\newpage
\section{Sog-Druckwellen eines Zuges}
	Ein vorbeifahrender Zug erzeugt ein Sog-Druck Feld. Dieses ist in der \autoref{fig:SogDruckFeld}
	dargestellt. Die roten Wellen bezeichnen den �berdruck, die blauen Wellen den Unterdruck. Das
	Sog-Druck Feld ist mit dem Zug verbunden und bewegt sich mit der gleichen Geschwindigkeit voran.
	Das Druckfeld nimmt mit der Zuggeschwindigkeit im Quadrat zu. Es ist umso kleiner, je gr�sser der
	Abstand vom Messpunkt zum Zug ist \cite{Niemann}.
	
	\begin{figure}
		\centering
		\includegraphics[width=15cm]{img/ZugDruckwelle.png}
		\caption[Typische Druck-Sog-Welle um einen Zug]{Typische Druck-Sog-Welle um einen Zug mit Geschwindigkeit $v_{Zug}$ aus \cite{Niemann}}
		\label{fig:SogDruckFeld}
	\end{figure}	
	Ein typischer Druckverlauf bei einem vorbeifahrenden Zug ist in der \autoref{fig:verlaufDruckwelle}
	gezeigt. Bei der Anfahrt des Zuges wird ein Druck-Sogwechsel gemessen, bei der Ausfahrt der Zuges
	einen Sog-Druckwechsel. Dort entstehen die gr�ssten Druckdifferenzen. Ebenfalls grosse
	Druckdifferenzen k�nnen bei den Kupplungsstellen der Zugzeile gemessen werden \cite{Niemann}.
	\begin{SCfigure}
		\centering
		\vspace{-0.5cm}
		\includegraphics[width=8cm]{img/VerlaufDruckwelleZug.png}
		\caption[Typische Druck-Sog-Welle um einen Zug]{Typischer Druckverlauf bei einem vorbeifahrenden Zug aus \cite{Niemann}}
		\label{fig:verlaufDruckwelle}
	\end{SCfigure}
	Der Abstand zwischen Durck- und Sogwelle L h�ngt nich von der Zugsgeschwindigkeit $v_{Zug}$ ab,
	jedoch vom Wandabstand zur Gleisachse $a_g$. 
	\newpage
	Der Zusammenhang zwischen Wandabstand $a_g$ und Abstand L wurde von Niemann et. Al \cite{Niemann} hergeleitet: 
	\begin{equation}
	L=6.9 \cdot (\frac{a_g}{3.8})^{0.65} 
	\end{equation}
	Bei zugnahen W�nden wird von L=6.9m ausgegangen. 
	Der zeitliche Abstand ist von $v_{Zug}$ abh�ngig und betr�gt bei zugnahen W�nden und der maximalen 
	Durchfahrtsgeschwindigkeit an Bahnh�fen von 200km/h 12ms. Je gr�sser der Wandabstand, desto gr�sser der Abstand 
	zwischen Durck- und Sogwelle und desto l�nger wird auch der zeitliche Abstand.  
\chapter{Beschleunigungssensoren}
	\section{Funktionsprinzip} \index{Beschleunigungssensor} 
	Beschleunigungssensoren werden gebraucht, um durch externe Kr�fte verursachte Beschleunigungen 
	zu messen. Nach Newtons zweitem Axiom (vgl. \autoref{sec:impuls}) berechnet sich die Beschleunigung z
	\begin{equation}
	\vec{a} = \frac{\vec{F}}{m}
	\end{equation}
	wobei die Richtung der Beschleunigung die Selbe wie die der Kraft ist. Um eine Beschleunigung zu 
	messen, muss man somit die einwirkende Kraft auf eine definierte Masse kennen. Ein wichtiger Teil 
	eines Beschleunigungssensors ist somit der Kraftsensor. Der Kraftsensor besteht aus einer Feder, 
	welche sich unter dem Einfluss der Kraft deformiert, und einem Deformationssensor, um den 
	Betrag der Deformation zu messen. 
	\begin{figure}
		\centering
		\captionsetup{width = 8cm}
		\includegraphics[width=8cm]{img/prinzip_accel.png}
		\caption{Prinzipieller Aufbau eines Beschleunigungssensors. Aus \cite{ModernSensor}}
		\label{fig:prinzip_accel}
	\end{figure}
	Die \autoref{fig:prinzip_accel} zeigt den prinzipiellen Aufbau eines Beschleunigungssensors. 
	Eine \textbf{Pr�fmasse} (engl. proof mass), auch seismische Masse genannt, wird durch eine Feder (engl. \index{Pr�fmasse} \index{D�mpfer} \index{Verschiebungssensor}
	spring) getragen. Ausserdem ist die Masse an einen \textbf{D�mpfer} (engl. damper) und einen
	\textbf{Verschiebungssensor} (engl. displacement sensor) gekoppelt. Der D�mpfer bremst die Bewegung der
	Pr�fmasse und der Verschiebungssensor misst die Ver�nderung der Lage der Pr�fmasse in Bezug auf die
	Ruhelage.
	
	\section{Eigenschaften}
	Beginnt eine Bewegung nach oben, dr�ckt die Pr�fmasse wegen ihrer Tr�gheit die Feder zusammen. 
	Die Feder wird durch den Steifigkeitsparameter $k$ charakterisiert. Die Feder dr�ckt gegen 
	die Pr�fmasse, so dass 
	\begin{equation}
	F=ma=k\Delta x= k(x_2-x_1)
	\end{equation}
	gilt. Somit ist die Verschiebung der Pr�fmasse 
	\begin{equation}
	x_2-x_1 = \frac{m}{k}a = Sa
	\end{equation}
	ein Mass f�r die Beschleunigung. Dabei ist $S$ die \index{Empfindlichkeit} \textbf{Empfindlichkeit} (engl. sensitivity) des Sensors und kann auch geschrieben 
	werden als 
	\begin{equation}
	S=\frac{m}{k}=\frac{1}{(\omega_0)^2} = \frac{1}{(2\pi f_0)^2}. 
	\end{equation}
	Dabei ist $f_0$ die \index{Resonanzfrequenz} \textbf{Resonanzfrequenz} des Systems. Der Sensor muss f�r genaue Messungen in einem 
	Frequenzbereich deutlich unter der Resonanzfrequenz betrieben werden. Um den Anwendungsbereich 
	zu vergr�ssern, muss die Pr�fmasse kleiner und die Feder steifer gew�hlt werden. Dies allerdings 
	hat den Effekt, dass der Sensor weniger empfindlich ist. Die Empfindlichkeit wird normalerweise
	bei einer Anregung mit einer harmonischen Schwingung bestimmter Frequenz gemessen. 
	
	Der \textbf{Frequenzgang} (engl. frequncy response) eines Sensors sollte im 
	benutzten Frequenzbereich konstant sein, so dass der Sensor f�r alle Frequenzen 
	gleich empfindlich ist. Im Datenblatt wird der Frequenzbereich angegeben, in welchem 
	die Empfindlichkeit nicht mehr als beispielsweise 5\% von der Referenzempfindlichkeit 
	abweicht. Die \autoref{fig:frequenzgang} zeigt einen Ausschnitt aus dem Datenblatt des piezoelektrischen 
	Beschleunigungssensors KS93. Dargestellt ist der typische Frequenzgang und dazugeh�rige 
	Frequenzen, bei welchen die Abweichung der Empfindlichkeit 3dB, 10\% und 5\% betr�gt. 
	\begin{figure}
		\centering
		\captionsetup{width = 8cm}
		\includegraphics[width=8cm]{img/frequenzgang.png}
		\caption{Typischer Frequenzgang des KS93 \cite{KS93}}
		\label{fig:frequenzgang}
	\end{figure}
	
	Der \textbf{Null-g-Output} (engl. zero stimulus output) ist der Ausgang des Sensors,  \index{Null-g-Output}
	wenn der Sensor nicht beschleunigt ist. Dies ist der Fall, wenn die empfindliche Achse
	des Sensors senkrecht zur Erdbeschleunigung gerichtet oder der Sensor im freien Fall ist. 
	
	Die \textbf{Bruchbeschleunigung} (engl. destruction limit) ist die maximale Beschleunigung, bei  \index{Bruchbeschleunigung}
	welchem der Sensor nicht zerst�rt wird. Dies ist f�r viele Applikationen eine kritische Eigenschaft. 
	
	\section{Technologien} \index{Mikrominiaturisierung} \index{MEMS}
	Ein Beschleunigungssensor gem�ss dem Aufbau vorgestellt in \autoref{fig:prinzip_accel} ist 
	auf viele verschiedene Arten realisierbar. Meistens ist es der Verschiebungssensor, welcher
	die Verschiebung der Pr�fmasse misst, oder der Wandler, welcher sich bei den verschiedenen
	Sensortypen �ndert. 
	
	Ein wichtiger Trend bei den Sensortechnologien ist die \textbf{Mikrominiaturisierung}. Eine Form
	der Mikrominiaturisierung ist die \textbf{MEMS-Technologie} (engl. micro-electro-mechanical
	systems). MEMS-Sensoren beinhalten elektronische und mechanische Komponenten. Meist sind ist dies
	eine Einheit, welche die Daten verarbeitet (Mikroprozessor) und mindestens ein Mikrosensor welcher eine
	mechanische Funktion aus�bt. Die elektrische Schaltung und die mechanischen Komponenten sind alle
	in einem Siliziumsubstrat integriert. Die Dimensionen der Komponenten liegen in der Gr�ssenordnung von
	Mikrometern. Die Vorteile der MEMS-Technologie sind tiefere Kosten, Zuverl�ssigkeit und die kleine
	Gr�sse.
	
	\subsection{Kapazitive Beschleunigungssensoren} \index{Beschleunigungssensor!kapazitiv} \index{Beschleunigung!statische}
		Bei kapazitiven Beschleunigungssensoren verursacht die Verschiebung der Pr�fmasse eine �nderung
		der Kapazit�t eines oder mehrerer Kondensatoren. Dies kann durch das Ver�ndern der �berlappenden
		Fl�che der Kondensatorplatten oder des Abstandes der Kondensatorplatten erreicht werden. Meist
		werden kapazitive Beschleunigungssensoren aus zwei festen Elektroden welche zusammen mit der
		Pr�fmasse in der Mitte zwei Kondensatoren bilden. Bewegt sich die Pr�fmasse unter Einwirkung einer
		Kraft zur einen Elektrode hin, so entsteht eine Differenz zwischen den beiden Kapazit�ten, welche
		der Auslenkung der Pr�fmasse und somit der Beschleunigung entspricht. Durch den
		Differenzkondensator k�nnen diverse St�rungen der Umgebung welche die Kapazit�t eines Kondensators
		�ndern eliminiert werden. Ein Besipiel eines kapazitiven Beschleunigungssensors mit MEMS-
		Technologie zeigt \autoref{fig:kap_mems}. Die Vorteile kapazitiver Beschleunigungssensoren ist die
		realisierbare hohe Empfindlichkeit, die M�glichkeit, Beschleunigungen mit der Frequenz null
		(statische Beschleunigungen) zu messen, die g�nstige Produktion und die Mikrominiaturisierung.
			\begin{SCfigure}
				\centering
				%\captionsetup{width = 8cm}
				\includegraphics[width=7cm]{img/kap_mems.png}
				\caption{Beispiel eines kapazitiven MEMS Beschleunigungssensors. Base und Cap bilden die festen
					Elektroden, $C_{mc}$ und $C_{mb}$ sind die beiden Kapazit�ten zwischen den Elektroden und der
					Pr�fmasse. Durch die Auf- oder Abbewegung der Pr�fmasse ver�ndern sich die Abst�nde $d_1$ und
					$d_2$, was die Differenz der beiden Kapazit�ten �ndert \cite{ModernSensor}.}
				\label{fig:kap_mems}
			\end{SCfigure}	
	  
	\subsection{Piezoresresistive Beschleunigungssensoren} \index{Beschleunigungssensor!piezoresistiv} 
		Piezoresistive Beschleunigungssensoren basieren auf dem Prinzip, die Belastung der Federn durch
		die Verschiebung der Pr�fmasse zu messen. Durch die mechanische Belastung oder Deformation des
		piezoresistiven Materials �ndert sich dessen elektrischer Widerstand. Piezoresistive Sensoren
		haben eine grosse Bandbreite, k�nnen statische Beschleunigungen messen und halten Schocks bis zu
		10'000g aus. Ausserdem ist auch dieses Wirkprinzip f�r Mikrominiaturisierung geeignet.
		
	\subsection{Piezoelektrsiche Beschleunigungssensoren} \index{Beschleunigungssensor!piezoelektrisch} \index{Piezoelektrischer Effekt}
		Der piezoelektrische Effekt ist die Verschiebung von Ladungen durch die Belastung oder Deformation des kristallinen 
		Materials. Um die Ladung zu messen, platziert man zwei Elektroden mit dem Piezokristall dazwischen. Dadurch entsteht
		ein Kondensator, dessen Dielektrikum der Piezokristall ist. Wird der Piezokristall belastet, so l�dt sich der 
		Kondensator. In piezoelektrischen Beschleunigungssensoren dr�ckt die Pr�fmasse direkt auf den
		Piezokristall. Es liegt keine Feder zwischen Pr�fmasse und Piezokristall, der Piezokristall 
		selber stellt die Feder dar. Da diese relativ steif ist, haben piezoelektrische Beschleunigungssensoren 
		von Natur aus eine hohe Resonanzfrequenz. Es gibt viele verschiedene M�glichkeiten, Pr�fmasse und 
		piezoelektrischer Sensor im Geh�use anzuordnen, wovon die \autoref{fig:piezoel_sens} einige zeigt. 
		\begin{figure}
			\centering
			\captionsetup{width = 12cm}
			\includegraphics[width=12cm]{img/piezoel_sens.png}
			\caption{Verschiedene Anordnungsm�glichkeiten von Pr�fmasse und piezoelektrischem Sensor im Geh�use \cite{ModernSensor}.}
			\label{fig:piezoel_sens}
		\end{figure}	
		Piezoelektrische Sensoren welche die Signalverarbeitung im Sensor integriert haben, heissen IEPE	\index{IEPE}
		(engl. integrated electronics piezo electric) Beschleunigungssensoren. Sie wandeln das
		Ladungssignal in eine Spannung um. So kann das Signal mittels langer Kabel ohne Problem �bertragen
		werden. Sensoren ohne integrierte Signalverarbeitung ben�tigen zus�tzlich einen Ladungsverst�rker. 
		Das �bermittelte Signal ist anf�llig auf St�rungen. Deshalb wird ein spezielles Kabel f�r die �bertragung
		zum Ladungsverst�rker verwendet. Der Vorteil von Sensoren ohne integrierte Signalverarbeitung ist eine
		h�here Temperaturfestigkeit. 
	\subsection{Thermische Beschleunigungssensoren} \index{Beschleunigungssensor!thermisch}
		Bei thermischen Beschleunigungssensoren gilt es zwei Wirkprinzipien zu unterscheiden: 
	
	
	
	
	
	(Faden, 2016, \cite[S.393]{ModernSensor})


\part{Kl�rung der Machbarkeit}
\chapter{Versuche mit piezoelektrischen Sensoren} \label{chap:piezo}
	In einem ersten Schritt wurden Versuche mit piezoelektrischen Sensoren durchgef�hrt. Der Versuchsaufbau 
	wird in \ref{chap:Versuchsaufbau} beschrieben. Es folgen drei Versuche: der erste hat das Ziel, 
	eine Vorstellung von zu messenden Beschleunigungssignalen bei Vandalismus zu geben. Der zweite 
	Versuch untersucht die beste Positionierung von Beschleunigungssensoren in einer Uhr. Der dritte Versuch
	soll zeigen, ob die Unterscheidung zwischen einer Sog-Druckwelle eines Zuges und Vandalismus 
	anhand von Beschleunigungssignalen m�glich ist. Die Versuche sind jeweils in drei Abschnitte geteilt. 
	Dies sind Vorgehen, Messresultate und Fazit.  

	\section{Versuchsaufbau mit piezoelektrsichen Sensoren} \label{chap:Versuchsaufbau}
	An der Versuchsuhr werden drei piezoelektrische Beschleunigungssensoren befestigt. Die Positionen 
	der Sensoren sind in der \autoref{fig:Sensorpositionen} ersichtlich.  
	
	\begin{SCfigure}
		\centering
		%\vspace{-1cm}
		\includegraphics[width=7cm]{img/PosSensoren.png}
		\caption{Sensorpositionen:\\\\
			1) An der Montageleiste der Uhr innen, nahe der Konsole (Channel 1, gr�n)\\\\
			2) Am Uhrwerk, welches am Ziffernblatt befestigt ist (Channel 2, rot)\\\\
			3) Am Rahmen der Uhr aussen (Channel 3, blau)}
		\label{fig:Sensorpositionen}
	\end{SCfigure}
	An den Sensoren �ndert sich je nach Beschleunigung die Ladung Q. Diese muss mit einem Ladungsverst�rker
	in eine Spannung umgewandelt und verst�rkt werden. Dann werden die drei Signale direkt an ein Oszilloskop 
	angeschlossen. Die aufgenommenen Beschleunigungssignale werden als Bild und als Daten f�r Matlab exportiert. 
	Auf dem Computer k�nnen die Messungen ausgewertet werden. 
	
	Verwendete Sensoren und Ger�te: 
	\begin{longtable}{p{3cm} p{5cm} p{3cm}} \toprule
		\textbf{Ger�t}	& \textbf{Bezeichnung} & \textbf{Serinummer} \\	
		\midrule
		\endhead
		\multicolumn{2}{l}{\emph{Fortsetzung auf n�chster Seite}}	\\ \bottomrule \endfoot \endlastfoot 
		Piezoelektrsiche Sensoren	&Miniatur-Beschleunigungsaufnehmer 	KS93&		\\ 
		Ladungsverst�rker			&Messverst�rker M68D3					&		\\
		Oszilloskop					&LeCroy									&		\\
		\bottomrule					 	
		\caption{Verwendete Sensoren und Ger�te} 
		\label{tab:SensorenGer�te}
	\end{longtable}
	
	Die auf dem Oszilloskop aufgenommenen Bilder werden als Matlab-Daten exportiert um Sie dann in Matlab
	zu verarbeiten. F�r die Verarbeitung der Daten wurden zwei Funktionen geschrieben: readOszData() und 
	volt2acc(). Erstere liest die exportierten Daten des Oszilloskopes f�r jeden Kanal ein und speichert 
	diese in einer Datenstruktur. Die zweite rechnet die gemessenen Spannungen in die Beschleunigung in 
	g um. Dabei kann die verwendete Verst�rkung am Ladungsverst�rker als Parameter mitgegeben werden. 
	\lstset{
		basicstyle=\small\ttfamily,
		language=Matlab,               	    % choose the language of the code
		numbers=left,                   % where to put the line-numbers
		stepnumber=1,                   % the step between two line-numbers.        
		numbersep=5pt,                  % how far the line-numbers are from the code
		backgroundcolor=\color{white},  % choose the background color. You must add \usepackage{color}
		showspaces=false,               % show spaces adding particular underscores
		showstringspaces=false,         % underline spaces within strings
		showtabs=false,                 % show tabs within strings adding particular underscores
		tabsize=2,                      % sets default tabsize to 2 spaces
		captionpos=b,                   % sets the caption-position to bottom
		breaklines=true,                % sets automatic line breaking
		breakatwhitespace=true,         % sets if automatic breaks should only happen at whitespace
		frameround=tttt,
		frame=single,
		%basicstyle=\small,
		title=\lstname,                 % show the filename of files included with \lstinputlisting;
	}
	\lstinputlisting{code/readOszData.m}
	\lstinputlisting{code/volt2acc.m}
		
\section{Ergebnisse mit piezoelektrischen Sensoren}
	\subsection{Zu erwartende Signale bei Vandalismus} \label{subsec:signale}
		\paragraph{Vorgehensweise}
		Unter Vandalismus wird die mutwillige Zerst�rung verstanden, bei der es um eine Zerst�rungslust
		ohne weiteren Zweck geht. So lautet die Definition von Vandalismus auch \flqq absichtliche
		grundlose Zerst�rung \frqq. Nach dem Sprayen (62.5\%, absolut 4726 F�lle im Jahr 2015 im Kanton
		Bern) ist das Ein- und Zerschlagen (8\%, absolut 605 F�lle im Jahr 2016 im Kanton Bern) die
		h�ufigste bekannte Vorgehensweise bei Vandalismus \cite{KSB}. Es wird davon ausgegangen, dass in
		den meisten F�llen von Vandalismus an Bahnhofsuhren mit Steinen oder �hnlichen Gegenst�nden auf
		die Uhr geworfen wird.  Die zu erwartenden Signale werden folglich mit zwei verschieden schweren 
		Stahlkugeln und einem Softball erzeugt und ausgewertet.

		\paragraph{Messresultate}
		Die maximal gemessenen Beschleunigungssignale ohne Scheibenbruch sind in \autoref{fig:max}
		abgebildet. Der Sensor am Uhrwerk misst einen maximalen Beschleunigungspeak von -660g! Im Spektrum
		in \autoref{fig:spect_all} sind zwei Frequenzen deutlich zu erkennen. Die eine liegt bei ca. 50Hz,
		die andere bei ca. 13Hz. Durch die Videoanalyse erkennt man, welche Teile der Uhr mit
		diesen Frequenzen schwingen. Dies ist zum einen die ganze Uhr, welche mit ca. 13Hz auf und ab
		schwingt. Zum anderen ist es das Ziffernblatt, welches mit ca. 50Hz schwingt. Diese beiden 
		Schwingungen sind in \autoref{fig:darst_frequ} dargestellt. Bei dieser Messung
		war nur eine Scheibe mit Ziffernblatt montiert. 
		\begin{SCfigure}
			\centering
			\vspace{-1cm}
			\includegraphics[width=7cm]{img/msg5/frequenzen.png}
			\caption{Die beiden stark vertretenen Frequenzen im Spektrum: Ziffernblatt ca. 50Hz, ganze Uhr ca. 13Hz}
			\label{fig:darst_frequ}
		\end{SCfigure}
		
		\begin{SCfigure}
			%\centering
			\hspace{-0.8cm}
			\vspace{-1cm}
			\includegraphics[width=10cm]{img/msg5/max.png}
			\caption{Beschleunigung bei Aufprall einer Stahlkugel mit einer Masse von 537g und aus einem Meter H�he. \\\\
			Gr�n: Sensor Rahmen innen \\
			Rot: Sensor Uhrwerk \\
			Blau: Sensor Rahmen aussen	}
			\label{fig:max}
		\end{SCfigure}
		\vspace{2cm}
		\begin{figure}
			\centering
			\captionsetup{width=14cm}
			\includegraphics[width=15.5cm]{img/msg5/spektrum.png}
			\caption{Spektren der Sensoren bei Aufprall einer Stahlkugel mit einer Masse von 537g und aus einem Meter H�he. \\\\
				Gr�n: Spektrum Sensor Rahmen innen \\
				Rot: Spektrum Sensor Uhrwerk \\
				Blau: Spektrum Sensor Rahmen aussen}
			\label{fig:spect_all}
		\end{figure}
		Beim Beschleunigungssignal gemessen am Uhrwerk sind die Frequenzen des Ziffernblattes (ca. 50Hz)
		und der ganzen Uhr (ca. 13Hz) stark vertreten. Bei jenem am Rahmen aussen ist die Frequenz der
		ganzen Uhr stark vertreten. Diese ist etwas gr�sser als am Uhrwerk. Das Beschleunigungssignal
		innen an der Uhr ist �ber alle Frequenzen deutlich kleiner. Dies l�sst sich dadurch erkl�ren, dass
		weiter weg von der Befestigung an der Wand die Auslenkung der Uhr gr�sser ist. Da die Periode der
		Schwingung an jedem Punkt dieselbe ist, ist die Geschwindigkeit weiter aussen gr�sser. Daraus
		folgt, dass auch die Beschleunigung gr�sser ist.
		
		\begin{SCfigure}
			\centering
			\vspace{-1cm}
			\includegraphics[width=10cm]{img/msg2/break.png}
			\caption{Beschleunigung bei Aufprall einer Stahlkugel mit einer Masse von 537g und aus zwei Meter H�he. Die Scheibe bricht 
					 in Folge des Aufpralls.}
			\label{fig:bruch}
		\end{SCfigure}
		
		Bei einem Bruch der Scheibe sind die gemessenen Beschleunigungssignale kleiner. Dies ist in
		\autoref{fig:bruch} zu erkennen. Der Stoss ist bei einem Bruch weniger elastisch. Ein Teil der
		kinetischen Energie wird durch die plastische Deformation in innere Energie umgewandelt. Die Kugel
		prallt mit einer kleineren Geschwindigkeit zur�ck als ohne Bruch. Dadurch ist der �bertragene
		Impuls kleiner. Die Folge sind kleinere Beschleunigungen der Uhr. Ein weiterer Grund f�r kleinere
		Beschleunigungssignale ist, dass die Kugel nicht in der Mitte der Scheibe aufgeprallte.
		
		\paragraph{Fazit}
		Durch schwache Schl�ge aus einem Meter H�he lassen sich Beschleunigungssignale gr�sser als 600g am
		Uhrwerk messen. Die Kugel hat beim Aufprall eine Geschwindigkeit von 4.4 m/s und eine Energie von
		5.3J. Ein Amateur- Handballer erreicht beim Werfen von einem Ball mit einer Masse von ca. 480g aus
		dem Stand Abwurfgeschwindigkeiten von �ber 20 m/s \cite{Gorostiaga}. Die bombierte Scheibe ging
		bereits bei einer Abwurfh�he von 2m und somit einer Aufprallgeschwindigkeit von 6.62m/s in die Br�che.
		Es darf somit mit einem Beschleunigungssignalen im Bereich der dargestellten Messungen gerechnet
		werden.
		
		Nicht nur die Amplitude ist abh�ngig von der Position des Sensors, sondern auch das Frequenzspektrum. 
		Am Uhrwerk ist neben der Eigenschwingung der ganzen Uhr auch die Eigenschwingung des Ziffernblattes
		sichtbar. 
	\newpage	
	\subsection{Beste Positionierung}
	
		\paragraph{Vorgehensweise}
		F�r die Ermittlung der besten Positionierung eines Beschleunigungssensors wurde die schwerere Stahlkugel von 537g aus einer H�he von 
		50cm an verschiedenen Stellen auf die Uhr fallen gelassen. Es wurde gleichzeitig an drei Positionen an der
		Uhr gemessen. Diese sind auch im \autoref{chap:Versuchsaufbau} ersichtlich:  
		\begin{itemize}
			\item An der Montageleiste der Uhr innen, nahe der Konsole (Channel 1, gr�n)
			\item Am Uhrwerk, welches am Ziffernblatt befestigt ist (Channel 2, rot)
			\item Am Rahmen der Uhr aussen (Channel 3, blau)
		\end{itemize}
		Die Kugel wurde an 4 Positionen der Uhr fallen gelassen: 
		\begin{itemize}
			\item In der Mitte der Scheibe, Resultat Messung \autoref{fig:mittig}
			\item Seitlich an der Scheibe, Resultat Messung \autoref{fig:seitlich}
			\item Aussen auf die Scheibe, Resultat Messung \autoref{fig:aussen}
			\item Innen auf die Scheibe, Resultat Messung \autoref{fig:innen}
		\end{itemize}
		
		\paragraph{Messresultate}
		Die gr�ssten Beschleunigungssignale sind am Sensor am Uhrwerk zu messen. Die Beschleunigung erreicht
		Werte �ber 200g. Die zweitgr�ssten Beschleunigungssignale  sind am Sensor am Rahmen aussen zu messen. Die Beschleunigung
		liegt in einem Bereich bis 100g.  Am Sensor innen sind die kleinsten Ausschl�ge messbar. Diese Beschleunigungen
		�berschreiten maximal 10g. Der Sensor am Uhrwerk detektiert bei Aufprallen weiter von der Mitte
		entfernt kleinere, aber im Vergleich zu den beiden anderen Sensoren dennoch grosse Messwerte. 
		
		\paragraph{Fazit}
		Unabh�ngig davon, wo die Kugel auf der Scheibe aufprallt, k�nnen am Uhrwerk die gr�ssten
		Beschleunigungssignale gemessen werden. Ein	weiterer Vorteil der Messung am Uhrwerk zeigt sich in
		der Unterscheidung zwischen einem Wurf und einer Druck-Sogwelle eines Zuges in
		\autoref{sec:Unterscheidung}: Im Signal am Uhrwerk ist die Eigenfrequenz der ganzen Uhr, aber auch
		die Eigenfrequenz des Ziffernblatts erkennbar. Die beste Position f�r den Sensor ist demnach am
		Uhrwerk.
			\begin{figure}[H]
			\hspace{-0.8cm}
				\begin{minipage}[hbt]{8cm}
					\centering
					\includegraphics[width=8.5cm]{img/msg2/mittig.png}
					\vspace{-1cm}
					\caption{Aufprall mittig}
					\label{fig:mittig}
				\end{minipage}
				\hfill
				\begin{minipage}[hbt]{8cm}
					\centering
					\includegraphics[width=8.5cm]{img/msg2/seitlich.png}
					\vspace{-1cm}
					\caption{Aufprall seitlich}
					\label{fig:seitlich}
				\end{minipage}
			\end{figure}
			\begin{figure}[H]
				\hspace{-0.8cm}
				\begin{minipage}[hbt]{8cm}
					\centering
					\includegraphics[width=8.5cm]{img/msg2/aussen.png}
					\vspace{-1cm}
					\caption{Aufprall aussen}
					\label{fig:aussen}
				\end{minipage}
				\hfill
				\begin{minipage}[hbt]{8cm}
					\centering
					\includegraphics[width=8.5cm]{img/msg2/innen.png}
					\vspace{-1cm}
					\caption{Aufprall innen}
					\label{fig:innen}
				\end{minipage}
			\end{figure}
	\clearpage	
	\newpage
	\subsection{Unterscheidung Druck-Sogwelle des Zuges und Wurf} \label{sec:Unterscheidung}
		\paragraph{Vorgehensweise}
			Die auf die Uhr einwirkenden Kr�fte unterscheiden sich bei einem Wurf oder einer Zugdruchfahrt
			prim�r durch folgende Faktoren:
			\begin{itemize}
				\item Gr�sse der Kontaktfl�che
				\item Dauer der Einwirkung
			\end{itemize}
			Da es sehr zeitaufw�ndig ist, eine Bewilligung f�r Messungen auf Perrons im Fahrleitungsbereich
			zu erhalten, musste eine andere Vorgehensweise gew�hlt werden. Wie in den obigen Abschnitten
			beschrieben, wurde mit zwei verschieden schweren Stahlkugeln Messungen gemacht. Um die
			Krafteinwirkung eines Zuges zu simulieren, wurden zus�tzlich Messungen mit einem Softball
			durchgef�hrt. Dieser hat eine �hnliche Masse (180g) wie die leichtere Stahlkugel (256g). 
			
		\paragraph{Messresultate}
			In der \autoref{fig:sb_k_time_vergl} sind im oberen Teilbild die Beschleunigungssignale 
			abgebildet, welche durch den Aufprall eines Softballs hervorgerufen werden. Im unteren Teilbild 
			sind die Beschleunigungssignale beim Aufprall einer Stahlkugel mit �hnlichem Gewicht dargestellt. 
			Dabei wurde das rote Signal am Uhrwerk, das blaue Signal am Rahmen aussen an der Uhr gemessen. 
			Es wird ersichtlich, dass die Amplitude der Beschleunigungssignale gemessen am Uhrwerk beim
			Aufprall eines Softballs mehr als um den Faktor zehn kleiner sind als beim Aufprall der Kugel.
			Es f�llt auf, dass beim Aufprall des Softballs die beiden gemessenen Signale am Uhrwerk und am 
			Rahmen aussen die gleiche Frequenz aufweisen. 
			\begin{SCfigure}
				\centering
				\vspace{-1cm}
				\includegraphics[width=10cm]{img/msg5/vergl_sb_k.png}
				\caption{Oberes Teilbild: 
						Softball 180g aus 1m H�he. Unteres Teilbild:\\ 
						Kugel 256g aus 1m H�he.\\\\
						Rot: Sensorsignal Uhrwerk\\
						Blau: Sensorsignal Rahmen aussen}
				\label{fig:sb_k_time_vergl}
			\end{SCfigure}

			Die am Signal im Zeitbereich beobachteten Eigenschaften k�nnen auch im Frequenzspektrum
			festgestellt werden. Dieses ist in \autoref{fig:spect_vergl} dargestellt. Rot ist das Spektrum
			des Beschleunigungssignals beim Aufprall einer Kugel mit einer Masse von 256g aus einem Meter
			H�he, blau das Spektrum des Beschleunigungssignals beim Aufprall eines Softballs mit einer Masse
			von 180g aus einem Meter H�he, gr�n das Spektrum des Beschleunigungssignals beim Aufprall eines
			Softballs mit einer Masse von 180g aus zwei Metern H�he. Die Kugel und der Ball wurden dabei jeweils
			auf die Mitte der Scheibe fallen gelassen, und das Beschleunigungssignal am Uhrwerk ausgewertet.
			
			Beim Aufprall der Kugel (rot) ist die Frequenz des Ziffernblatt von ca. 50Hz am st�rksten
			vertreten. Diese Frequenz ist um den Faktor 5 st�rker vertreten als die Frequenz der ganzen Uhr
			von ca. 16Hz. Im Gegensatz dazu ist beim Aufprall des Softballs (gr�n) die Frequenz des Ziffernblattes
			um den Faktor 3 schw�cher vertreten als jene der ganzen Uhr. Der Aufprall der Kugel regt somit das
			Ziffernblatt st�rker an als der Softball.  
			
			\begin{figure}
				\centering
				\captionsetup{width=14cm}
				\includegraphics[width=15.5cm]{img/msg5/vergl_sek_sb_k_w.png}
				\caption{Spektren der Signale vom Sensor am Uhrwerk: \\\\
					Blau: Spektrum Sensorsignal bei Aufprall des Softballs 188g aus 1m H�he\\
					Rot: Spektrum Sensorsignal bei Aufprall der Kugel 256g aus 1m H�he \\
					Gr�n: Spektrum Sensorsignal bei Aufprall des Softballs 188g aus 2m H�he
					}
				\label{fig:spect_vergl}
			\end{figure}
			
			\paragraph{Fazit}
			
			L�sst man den Softball oder die Stahlkugel aus der gleichen H�he auf die Uhr prallen, haben beide
			Massen beim Aufprall ungef�hr die gleiche kinetische Energie. Die Messungen zeigen jedoch, dass durch den
			Aufprall der Stahlkugel das Ziffernblatt st�rker angeregt wird als durch den Aufprall des
			Softballs.
			
			Aus diesen Beobachtungen folgt die Hypothese, dass der Aufprall eines Mediums auf die Uhr zwei Effekte hat: 
			\begin{itemize}
				\item Verbiegung der ganzen Uhr
				\item Einbuchtung (Deformation) in der Scheibe
			\end{itemize}
			
			Da die Stahlkugel neben der Verbiegung der ganzen Uhr eine Einbuchtung der Scheibe zur Folge hat, 
			wird das Ziffernblatt st�rker angeregt. Diese Anregung wird durch die komprimierte Luft zwischen 
			Scheibe und Ziffernblatt �bertragen. 
			
			Im Folgenden sind je vier Bilder aus den Highspeedaufnahmen des Aufpralls einer Stahlkugel und 
			eines Softballs dargestellt. 
%			Die rote Linie zeigt den Schatten auf der Uhr, wessen Kante sich 
%			durch die Einbuchtung der Kugel verformt. Beim Aufprall des Softballs ver�ndert sich die 
%			Schattenkante nicht. 
			%\clearpage
			%\newpage
			\begin{SCfigure}[]
				\begin{minipage}[hbt]{5cm}
					\centering
					\includegraphics[width=5cm]{img/aufprallVergl/K1_r.png}
					\vspace{-1cm}
					%\caption{Aufprall mittig}
					%\label{fig:mittig}
				\end{minipage}
				\hfill
				\begin{minipage}[hbt]{5cm}
					\centering
					\includegraphics[width=5cm]{img/aufprallVergl/B1_r.png}
					\vspace{-1cm}
					%\caption{Aufprall seitlich}
					%\label{fig:seitlich}
				\end{minipage}
			\end{SCfigure}
			\begin{SCfigure}[]
				\begin{minipage}[hbt]{5cm}
					\centering
					\includegraphics[width=5cm]{img/aufprallVergl/K2_r.png}
					\vspace{-1cm}
					%\caption{Aufprall mittig}
					%\label{fig:mittig}
				\end{minipage}
				\hfill
				\begin{minipage}[hbt]{5cm}
					\centering
					\includegraphics[width=5cm]{img/aufprallVergl/B2_r.png}
					\vspace{-1cm}
					%\caption{Aufprall seitlich}
					%\label{fig:seitlich}
				\end{minipage}
			\end{SCfigure}
			\begin{SCfigure}[]
				\begin{minipage}[hbt]{5cm}
					\centering
					\includegraphics[width=5cm]{img/aufprallVergl/K3_r.png}
					\vspace{-1cm}
					%\caption{Aufprall mittig}
					%\label{fig:mittig}
				\end{minipage}
				\hfill
				\begin{minipage}[hbt]{5cm}
					\centering
					\includegraphics[width=5cm]{img/aufprallVergl/B3_r.png}
					\vspace{-1cm}
					%\caption{Aufprall seitlich}
					%\label{fig:seitlich}
				\end{minipage}	
			\end{SCfigure}
			
			\begin{SCfigure}[]
				\begin{minipage}[hbt]{5cm}
					\centering
					\includegraphics[width=5cm]{img/aufprallVergl/K4_r.png}
					\vspace{-1cm}
					%\caption{Aufprall mittig}
					%\label{fig:mittig}
				\end{minipage}
				\hfill
				\begin{minipage}[hbt]{5cm}
					\centering
					\includegraphics[width=5cm]{img/aufprallVergl/B4_r.png}
					\vspace{-1cm}
					%\caption{Aufprall seitlich}
					%\label{fig:seitlich}
				\end{minipage}		
		\end{SCfigure}
		
		Der Softball wird beim Aufprall zusammengestaucht und hat eine gr�ssere Kontaktfl�che mit der Scheibe als beim Aufprall
		der Stahlkugel. In der Scheibe entsteht dadurch keine sichtbare Einbuchtung. Dadurch wird die Luft zwischen Scheibe und Ziffernblatt weniger
		komprimiert und das Ziffernblatt dadurch weniger angeregt. 
		
		Die auf die Scheibe dr�ckende Luft bei einem vorbeifahrenden Zug greift auf der ganzen Scheibe gleichm�ssig an. Es wird
		daher angenommen, dass das Ziffernblatt dadurch noch weniger als beim Softball angeregt wird. 
		
		Highspeedaufnahmen verschiedener Bahnhofsuhren bei zwei vorbeifahrenden Personenz�gen und einem G�terzug haben keine Bewegung der ganzen Uhr 
		gezeigt. Die Geschwindigkeit der Z�ge betrug ungef�hr 80km/h. Es wurde am Bahnhof Aarburg-Oftringen und Rothrist gemessen. 
		Diese Resultate best�tigen weiter, dass eine Unterscheidung zwischen Vandalismus und vorbeifahrendem Zug machbar ist. 
		
		
	\begin{tikzpicture}[every node/.style={draw,outer sep=0pt,thick}]
		\tikzstyle{spring}=[thick,decorate,decoration={zigzag,pre length=0.3cm,post length=0.3cm,segment length=6,amplitude=1.5mm}]
		\tikzstyle{damper}=[thick,decoration={markings,  
			mark connection node=dmp,
			mark=at position 0.5 with 
			{
				\node (dmp) [thick,inner sep=0pt,transform shape,rotate=-90,minimum width=15pt,minimum height=3pt,draw=none] {};
				\draw [thick] ($(dmp.north east)+(2pt,0)$) -- (dmp.south east) -- (dmp.south west) -- ($(dmp.north west)+(2pt,0)$);
				\draw [thick] ($(dmp.north)+(0,-5pt)$) -- ($(dmp.north)+(0,5pt)$);
			}
		}, decorate]
		\tikzstyle{ground}=[fill,pattern=north east lines,draw=none,minimum width=1cm,minimum height=0.3cm]
		
		\node (M1) [minimum width=2cm, minimum height=2.5cm,draw,outer sep=0pt,thick] {$m1$};				% Masse 1
		
		\node (wall) [ground, rotate=-90, minimum width=4cm,yshift=2.5cm] {};								% Mauer
		\draw[thick] (wall.south east) -- (wall.south west);												% Strich an Mauer
		
		\draw [spring] (wall.south) -- node[above=2mm,draw=none]{$k_1$}($(M1.south east)!(wall.100)!(M1.north east)$);	% Befestigung Mauer
		
		\node (M2) [minimum width=0.5cm, minimum height =1cm,yshift=-0.75cm, xshift=-2.5cm] {$m2$};
		%\draw [spring](M2.east) -- node[below=2mm,draw=none]{$k_4$} ($(M1.north west)!(M2.west)!(M1.south west)$);		% 
		
		\node (M3) [minimum width=0.5cm, minimum height =2.5cm,yshift=0cm, xshift=-4.5cm] {$m3$};
		\draw [spring] (M2.west) -- node[below=2mm,draw=none]{$k_2$}($(M3.north east)!(M2.east)!(M3.south east)$);		% Komprimierte Luft
		\draw [spring] (M3.60) -- node[above=2mm,draw=none]{$k_3$}($(M1.north west)!(M3.60)!(M1.south west)$);
		
		
		\draw [latex-, thick] (M3.west) ++ (-0.2cm,0) --node[above=2mm,draw=none]{$F$} +(-1cm,0);
		\node [rectangle,text width=6.5cm,align=left, draw=none] at ( $ (current bounding box.center) + (8,0) $ ) {$m_1$: Rahmen, Montageleiste\\ $m_2$: Ziffernblatt \\$m_3$: Scheibe \\ $k_1$: Befestigung an Wand\\ $k_2$: komprimierte Luft\\ $k_3$: Kraft�bertragung �ber Scheibenrand\\ };		
		
	\end{tikzpicture}
		
	
\chapter{Versuche mit dem MEMS-Sensor} 
	In einem weiteren Schritt soll ein MEMS-Sensor in Betrieb genommen werden, welcher im Produkt angewendet wird. 
	\section{Versuchsaufbau}
	Die Versuche mit den piezoelektrischen Sensoren \ref{subsec:signale} haben gezeigt, dass
	Beschleunigungen �ber 100g gemessen werden. Deshalb wurde besonders auf einen grossen
	Dynamikbereich geachtet. Ausserdem muss die Abtastfrequenz �ber 100Hz liegen. Es wurde der Sensor
	H3LIS331DL von STMicroelectronics eingesetzt. Dieser hat einen einstellbaren Bereich von $\pm100g$,
	$\pm200g$ und $\pm400g$. Die Daten werden digitalisiert. Es kann �ber I2C oder SPI mit dem Sensor
	kommuniziert werden. Die Abtastfrequenz kann zwischen 0.5Hz bis 1kHz konfiguriert werden. Der
	Sensor ist mit der Gr�sse von 3x3x1mm$^2$ sehr klein. Zwei Interruptausg�nge k�nnen konfiguriert
	werden. Eine M�glichkeit ist beispielsweise, dass ein �berschreiten eines Schwellwerts angezeigt
	wird. Weitere Eigenschaften des Sensors k�nnen dem Datenblatt \cite{H3L} entnommen werden.
	
	Als zweite Komponente wurde ein FRDM-K64F Freedom Board verwendet. Dieses bietet neben der 
	Konfiguration des Sensors die M�glichkeit, die ausgelesenen Daten auf einer SD-Karte zu speichern. 
	
	Mit einer Power Bank kann das Freedom Board und der Sensor gespeist werden. So bleibt die Messeinheit
	mobil. 
	\begin{figure}
		\centering
		%\vspace{-1cm}
		\includegraphics[width=15cm]{img/mems_aufbau.png}
		\caption{Der Aufbau mit MEMS-Sensor, Freedom Board und Power Bank}
		\label{fig:mems_aufbau}
	\end{figure}
	
	\subsection{Aufbau der Software}
	F�r die Software wurde FreeRTOS verwendet. Die Software ist in drei Tasks aufgebaut. Der Idle-Task
	wird immer dann ausgef�hrt, wenn kein anderer Task die CPU ben�tigt. Der Main-Task hat die Aufgabe,
	die Tasten zu pollen und die Events zu handeln. Wichtiger f�r den Aufbau der Software ist der
	Sensor-Task und der SD-Karten-Task. Der Sensor-Task kommuniziert �ber I2C mit dem Sensor und liest
	Daten, falls vorhanden, vom Sensor aus. Er speichert ausserdem diese Daten in einer Queue. Der
	SD-Karten-Task wiederum liest die Daten aus der \index{Queue}, speichert diese in einem Buffer als String.
	Ist der Buffer voll, wird dieser auf die SD-Karte gespeichert.
	
	Im Sensor-Task und im SD-Karten-Task werden Zustandsautomaten durchgearbeitet. Diese k�nnen einerseits
	durch Tastendr�cke oder Interne Signale beeinflusst werden. Die Kommunikation zwischen dem Main-Task,
	dem Sensor-Task und dem SD-Karten-Task erfolgt dabei mittels Task Notifications. 
	
	Task Notifications gibt es ab FreeRTOS-Version V8.2.0 (16. Januar 2015). Sie dienen dienen der
	Inter-Task-Kommunikation und Synchronisation, �hnlich den Semaphoren. Jeder Task hat dabei einen
	32-bit notification value. In dieser Anwendung wurde jeweils ein Bit des notification values einem
	Tastendruck oder einem internen Signal zugewiesen. Der Main-Task setzt beispielsweise bei einem
	Tastendruck des Button 2 das erste Bit im notification value. Der Sensor-Task wechselt seinen
	Zustand von "'IDLE"' in "'MEASURE"', wenn dieses Bit gesetzt wurde. Task Notifications sind
	effizienter als Semaphoren \cite{TaskNotification}. 

	Die \autoref{fig:sm_sensor} und \autoref{fig:sm_sd} veranschaulichen die beiden Zustandsautomaten. 
	\begin{figure}
		\centering
		%\vspace{-1cm}
		\includegraphics[width=15cm]{img/sm_sensor.png}
		\caption{Zustandsdiagramm Sensor-Task}
		\label{fig:sm_sensor}
	\end{figure}
	
	Der Zustandsautomat durchl�uft den Zustand "'INIT ACCEL"' in welchem der Sensor konfiguriert 
	wird. Zuerst wird getestet, ob der Sensor antwortet. Dann werden der Messbereich, die Abtastrate, 
	der Power-Modus und die Art der Datenaufbereitung eingestellt. Der Sensor kann im Normal Powermodus
	betrieben werden, oder aber in verschiedenen Low Powermoden. In den Low Powermoden ist die 
	Abtastrate nicht konfigurierbar und tiefer als im Normal Powermodus. Die Beschleunigungsdaten des Sensors werden 
	in zwei 8-bit Registern abgelegt. 
	
	Nachdem der Sensor konfiguriert ist, wechselt der Zustandsautomat in den Zustand "STARTUP". Hier wird
	so lange gewartet, bis der SD-Karten-Task mit einer Task Notification bekannt gibt, dass er das File 
	System gemounted hat. 
	
	Der Zustandsautomat im Sensor-Task hat grunds�tzlich zwei Schlaufen. Zum einen kann durch einen 
	Tastendruck des Buttons 2 die Messung gestartet w
	\begin{figure}
		\centering
		%\vspace{-1cm}
		\includegraphics[width=12cm]{img/sm_sd.png}
		\caption{Zustandsdiagramm SD-Karten-Task}
		\label{fig:sm_sd}
	\end{figure}
\chapter{Fazit der Machbarkeitsstudie}
	Ein Schlag auf die Scheibe der Aussenuhr hat schnell grosse Beschleunigungen im dreistelligen
	Bereich zur Folge. Die beste Position f�r den Beschleunigungssensor ist am oder im Uhrwerk.
	Unabh�ngig vom der Position des Schlags auf der Scheibe werden dort die gr�ssten Beschleunigungen
	gemessen. Der Schlag auf die Scheibe hat neben einer Verbiegung der ganzen Uhr eine Einbuchtung der
	Scheibe und damit eine Komprimierung der Luft zwischen Scheibe und Ziffernblatt zur Folge. Deshalb
	wird das Ziffernblatt angeregt, an welchem das Uhrwerk und somit der Beschleunigungssensor
	befestigt ist. Im Gegensatz dazu wird durch den Aufprall eines Softballs fast
	ausschliesslich die Uhr verbogen. Dies ist auf die Gr�sse der Kontaktfl�che bei der Kraft�bertragung zwischen
	Ball und Scheibe zur�ck zu schliessen.
	Der am Uhrwerk platzierte Sensor erf�hrt durch die fehlende Einbuchtung der Scheibe eine um
	Faktoren geringere Beschleunigung. Da die Sog-Druckwelle des Zuges auf der ganzen Fl�che
	angreift, wird erwartet, dass keine Einbuchtung der Scheibe zustande kommt. Zus�tzlich ist die
	Dauer der Anregung durch den vorbeifahrenden Zug zu lang, um die Uhr in Schwingung zu versetzen.
	
	Die Inbetriebnahme des Datenloggers mit dem MEMS-Sensors H3LIS331DL von STMicroelectronics hat gezeigt, dass 
	Schock Detektion mit einem g�nstigen Sensor und einfachem Algorithmus realisierbar ist. Ein Schwellwert gen�gt, 
	um den Schock zu detektieren. Abschliessende 
	Tests der Schock Detektion an der Aussenuhr haben die Machbarkeit auch im Versuch best�tigt. 



\part{Umsetzung}
\chapter{Realisierung}
	Zus�tzlich zur Abkl�rung, ob eine Schock Detektion m�glich ist, sollen verschiedene Ideen f�r die 
	Realisierung einer solchen zusammengetragen werden. Vorgegeben ist dabei der Aufbau des Clock Set, beschrieben 
	in \autoref{sec:clockset}. Die verschiedenen Realisierungsideen werden in \autoref{sec:ideen}
	aufgelistet und die jeweiligen Vor- und Nachteile diskutiert. In \autoref{sec:modul} wird die im Rahmen 
	dieser Arbeit umgesetzte Idee vorgestellt. 
	
	\section{Aufbau des Clock Sets} \label{sec:clockset}
		\vspace{-0.75cm}
		\begin{figure}	
			\centering
			\captionsetup{width=8cm}
			
			\begin{tikzpicture}
			\node[above right] (img) at (0,0) {\includegraphics[width=8cm]{img/clockset.png}};
			
			\draw[line width=1pt]				
			(1.2, 5) node[left=1mm] {Uhrwerk}
			-- (3.6, 3.3);
			\fill (3.6, 3.3) circle (2pt);
			
			\draw[line width=1pt]
			(6,1.5) node[below=1mm] {Clock Controller}
			-- (5.0, 3.7);
			\fill (5.0, 3.7) circle (2pt);
			
			\draw[line width=1pt]
			(1.6,5.5) node[above=1mm] {LED-Beleuchtung}
			-- (2.9, 4.7);
			\fill (2.9, 4.7) circle (2pt);
			
			\draw[line width=1pt]
			(4,0.5) node[right=1mm] {Ziffernblatt und Zeiger}
			-- (3,1.6);
			\fill (3,1.6) circle (2pt);
			\end{tikzpicture}
			\caption[Aufbau des Clock Set]{Aufbau des Clock Set, Ausschnitt aus Datenblatt \cite{ClockSet}}
			\label{fig:clockset}
		\end{figure}
	Der Aufbau des Clock Set besteht aus einem Uhrwerk, einer LED-Beleuchtung, einem Clock Controller
	und bis zu zwei Uhrwerken mit Ziffernblatt und Zeigern \cite{ClockSet}. Die \autoref{fig:clockset}
	zeigt die genannten Komponenten. Das Uhrwerk immer das Selbe. 
	Der Clock Controller variiert je Synchronisationsart des Uhrwerkes (DCF, GPS, Mobaline, NTP) und der
	Speisungsart. Auf dem Uhrwerk und dem Clock Controller ist jeweils ein ATMEL- Kontroller best�ckt.
	 Ein Zweidrahtbus verbindet
	den Clock Controller und das Uhrwerk. Dabei ist die Kommunikation zwischen den beiden Komponenten grunds�tzlich 
	unidirektional, vom Clock Controller zum Uhrwerk. Die Schnittstelle zum Netzwerk stellt der Clock 
	Controller her. 

	
	\section{Realisierungsideen} \label{sec:ideen}
	Es gibt verschiedene M�glichkeiten, wie ein Shock Detector in den bestehenden Aufbau des Clock Set
	integriert werden kann. Diese haben jeweils verschiedene Vor- und Nachteile. In der
	\autoref{tab:ideen_realisierung} sind f�nf Ideen f�r die Realisierung mit den jeweiligen Vor- und
	Nachteilen aufgelistet.
	
	Der Sensor wird gem�ss \autoref{subsec:bestPos} am Uhrwerk positioniert, da dort die gr�ssten 
	Beschleunigungssignale und die f�r die Unterscheidung zwischen Zugdurchfahrt und Shock Event 
	wichtige Eigenfrequenz des Ziffernblattes gemessen werden kann. Der Sensor kann somit entweder
	auf der Platine des Uhrwerks oder auf einem externen Print (folgend auch Modul genannt) platziert werden. 
	%\newpage
	Ein zus�tzliches Modul wird an der R�ckseite des Uhrwerkgeh�uses geklebt oder geschraubt. 
	Bei einer doppelseitigen Uhr bleiben ungef�hr 70mm Platz zwischen den beiden Uhrwerken
	(METRO-Uhr mit zwei SEM 100t Uhrwerken). Ein externes Modul sollte somit eine maximale 
	H�he von 30mm nicht �berschreiten.  
	
	F�r die Konfiguration und das Auslesen der Beschleunigungsdaten bieten sich drei M�glichkeiten an. 
	Der Mikrokontroller auf dem Uhrwerk und jener auf dem Clock Controller haben beide I2C, SPI und UART.
	Auf beiden Mikorkontrollern l�uft ein Betriebssystem. Jenes auf dem Uhrwerk ist mit den bestehenden Tasks
	st�rker ausgelastet als das Betriebssystem auf dem Clock Controller. 
	Die dritte M�glichkeit ist ein zus�tzlicher Mikrokontroller welcher als einzige Aufgabe den 
	Sensor konfiguriert, ausliest und mit dem Clock Controller oder dem Uhrwerk kommuniziert. 
	
	
	In jedem Fall muss das Shock Event Signal �ber den Clock Controller abgesetzt werden, 
	da dies die Schnittstelle zum Netzwerk ist. Wird der Sensor mit dem ATMEL Kontroller
	auf dem Uhrwerk oder dem externen Modul ausgewertet, w�re eine Eindrahtverbindung 
	denkbar, welche ein Shock Event signalisiert. Soll jedoch die Auswertung mit dem 
	ATMEL Kontroller auf dem Clock Controller erfolgen, so muss I2C oder SPI mittels einer
	Kabelverbindung zum Clock Controller gef�hrt werden. Typischerweise wird I2C oder SPI 
	bei on-board Anwendungen verwendet, d.h. innerhalb einer Platine. Bei l�ngeren Kabelverbindungen 
	wird der Kommunikationskanal schlechter und die Bitfehlerrate erh�ht sich. Bei der Platzierung
	des Sensors auf der Platine im Uhrwerk in Kombination mit einer Kabelverbindung auf den Clock Controller ist eine 
	Ab�nderung des Geh�uses des Uhrwerkes n�tig, damit die Kabel hinausgef�hrt werden k�nnen. 
	Die Ab�nderung der Gusswerkzeuge kann mehrere tausend Franken kosten. Die M�glichkeit, 
	ein Shock Event Signal �ber den bestehenden Zweidrahtbus abzusetzen, m�sste untersucht werden. 
	Dies bedeutet weiteren Entwicklungsaufwand. 
	
	
	Gemeinsam mit der Firma Moser-Baer AG wurde entschieden, in einem ersten Schritt die 
	Realisierungsidee Nr. 4 zu realisieren. Diese erm�glicht eine hohe Flexibilit�t um weiter
	Messungen durchzuf�hren. Es ist denkbar, bei Kundenw�nschen das Modul bereits einzusetzen. 
	Sollte sich der Shock Detector bew�hren, kann dieser in einem weiteren Schritt weiter in den 
	bestehenden Aufbau integriert werden. 
	
	\newgeometry{
		left=2cm,
		right=2cm,
		top=1cm,
		bottom=1cm,
		bindingoffset=5mm
	}	
\begin{landscape}
	\begin{longtable}{p{0.5cm} p{6cm} p{2.5cm} p{6cm} p{8cm}} \toprule
		\textbf{Nr.}	& \textbf{Beschreibung} & \textbf{Ben�tigte Komponenten} 	& \textbf{Pro} 	& \textbf{Contra} \\	
		\midrule
		\endhead
		\multicolumn{2}{l}{\emph{Fortsetzung auf n�chster Seite}}	\\ \bottomrule \endfoot \endlastfoot 
		1 & Sensor auf Platine Uhrwerk, 
			Auswertung auf ATMEL Uhrwerk, 
			Signal �ber Mobaline an Clock 
			Controller							& Sensor 							& $\bullet$ Elegante L�sung da 
																					  wenige Komponenten ben�tigt
																					  $\Rightarrow$ g�nstig\newline	
																					  $\bullet$ keine zus�tzliche Montage\newline
																					  $\bullet$keine Kabelverbindung I2C oder SPI		& $\bullet$ Auslastung ATMEL Uhrwerk bereits hoch 
																																		  $\Rightarrow$ Auswertung mit Sensorregistern\newline
																																		  $\bullet$ zwei Hardwareversionen des Uhrwerks\newline
																																		  $\bullet$ Hardware�nderung Uhrwerk\newline 
																																		  $\bullet$ ev. Konflikt auf Mobaline $\Rightarrow$ 
																																		  zus�tzlicher Entwicklungsaufwand \\ \midrule 
		2 & Sensor auf Platine Uhrwerk, 
			Auswertung auf zus�tzlichem 
			Kontroller, Signal �ber Mobaline
			an Clock Controller					& Sensor und Kontroller				& $\bullet$ relativ wenig Komponenten \newline
																					  $\bullet$ wie in Nr. 1 Pkt. 2 \& 3
																									  									& $\bullet$ zus�tzlicher Kontroller $\Rightarrow$ teurer als Nr. 1 \newline
																																		  $\bullet$ wie in Nr. 1 Pkt. 2-4 \\ \midrule		
		3 & Sensor auf Modul extern an Uhrwerk, 
			I2C / SPI Kabelverbindung auf Clock-
			Controller, Auswertung auf Clock-
			Controller							& Sensor, zus�tzliche Platine, 
												  Montagematerial					& $\bullet$ keine Hardware�nderung \newline
																					  $\bullet$ nicht �ber Mobaline \newline				
																					  $\bullet$ flexibel einbaubar						& $\bullet$ Kabelverbindung I2C/SPI\newline
																																		  $\bullet$ zus�tzlicher Montageaufwand \newline
																																		  $\bullet$ Speisung �ber Kabel \newline
																																		  $\bullet$ mehr Komponenten $\Rightarrow$ teurer \\ \midrule
		4 & Sensor auf Modul extern an Uhrwerk, 
			zus�tzlicher Kontroller auf Modul 
			f�r die Auswertung					& Sensor, Kontroller, zus�tzliche 
												  Platine, Montagematerial			& $\bullet$ wie Nr. 3 \newline
																					  $\bullet$ keine Kabelverbindung I2C oder SPI \newline
																					  $\bullet$ flexibel f�r weitere Experimente und 
																					  Messungen											& $\bullet$ wie in Nr. 3 Pkt. 2-4 \\ \midrule
		5 & Sensor auf Layout Uhrwerk, 
			Kabelverbindung I2C / SPI auf 
			Clock Controller 					& Sensor 							& $\bullet$ keine zus�tzliche Montage \newline
																					  $\bullet$ wenig Komponenten						& $\bullet$ zwei Hardwareversionen des Uhrwerks \newline
																																		  $\bullet$ Kabelverbindung I2C / SPI \newline
																																		  $\bullet$ Ab�nderung Geh�use wegen zus�tzlicher Kabelf�hrung 
																																		  $\Rightarrow$ hohe Kosten f�r Gusswerkzeugab�nderung \\ \bottomrule			 	
		\caption{Ideen f�r die Ralisierung des Shock Detectors} 
		\label{tab:ideen_realisierung}
	\end{longtable}%
\end{landscape}
\restoregeometry
	\newpage

	\section{Shock Detector Modul}	\label{sec:modul}
	Die Realisierung des Shock Detectors erfolgt gem�ss der Realisierungsidee Nr. 4 in der
	\autoref{tab:ideen_realisierung}. Der Sensor wird auf einer externen Platine mit einem
	Mikrokontroller konfiguriert und ausgelesen. Zus�tzlich zu den erw�hnten Komponenten in der Realisierungsidee
	Nr. 4 soll eine Anbindung an den Computer realisiert werden. So kann das Modul konfiguriert werden
	und Beschleunigungsdaten aufgezeichnet werden. 
	
	Um das Modul am Uhrwerk zu montieren, sind zwei M�glichkeiten vorgesehen. Einerseits sollen nur SMD
	Bauteile verwendet werden, so dass der Sensor an das Uhrwerk geklebt werden kann. Der Vorteil davon
	ist, dass dazu das Geh�use des Uhrwerks nicht modifiziert werden muss. Andererseits sind vier L�cher
	im Print, um das Modul am Uhrwerk anzuschrauben.
	
	\paragraph{Schema}
	\autoref{fig:schema} zeigt das Schema des Shock Detector Moduls. Grunds�tzlich wurde darauf geachtet, 
	geeignete Bauteile aus dem Lagerbestand der Moser-Baer AG zu w�hlen. Andernfalls wurden m�glichst
	g�nstige Bauteile bestellbar bei einem bekannten Distributor wie Mouser oder Distrelec verwendet. 
	Die Datenbl�tter der verwendeten Bauteile sind im Literaturverzeichnis mit einer Verlinkung 
	auf eine Onlineversion aufgelistet und zus�tzlich auf der CD.
	
	Es wird der gleiche Mikrokontroller wie auf dem Uhrwerk eingesetzt, ATxmega32A4U \cite{Atmel} von
	ATMEL. Dieser wird mit einem externen Quarzoszillator mit einer Frequenz von 12MHz betrieben. Der
	Mikrokontroller wird �ber die Programmierschnittstelle PDI programmiert. Der Sensor wird �ber I2C
	angesprochen. Der Sensor kann aus der Produktreihe der High-g Beschleunigungssensoren von
	STMicroelectronics gew�hlt werden. So kann der im Versuch verwendete  H3LIS100DLTR oder auch der
	pinkompatible und halb so teure H3LIS200DLTR best�ckt werden.
	
	Die Anbindung an den Computer wird mit dem USB zu UART Interface FT232B \cite{FT232B} von FTDI realisiert. Dieses
	IC kann verschieden gespeist werden. Die beim Shock Detector Modul verwendete Konfiguration ist Bus
	Powered, d.h. von der USB Schnittstelle her gespeist. 
	
	F�r die Speisung des Shock Detection Moduls sind zwei M�glichkeiten vorgesehen. Einerseits kann
	vom Clock Controller die 3.3V Speisung an den Anschlussblock X2 angeschlossen werden. Andererseits
	kann das Modul �ber die USB Schnittstelle gespeist werden. Mit einem Jumper wird die jeweilige
	Speisungsart konfiguriert. Da der Beschleunigungssensor mit einer Speisespannung von 2.16 bis 3.6V
	und der ATMEL Mikrokontroller mit 1.6 bis 3.6V arbeitet, wird bei Speisung �ber USB einen
	zus�tzlichen Spannungsregler ben�tigt. Der FT232B hat intern einen eigenen Spannungsregler, dieser
	kann jedoch maximal 5mA liefern und ist daher f�r diese Anwendung nicht geeignet. Es wird deshalb
	der lineare low-dropout (LDO) Spannungsregler LM2936 \cite{LM2936} mit einer fixen Ausgangsspannung
	von 3V verwendet.
		
	Verschiedene LEDs zeigen den Zustand des Shock Detecotor Moduls an. Die LED D4 zeigt an, ob die USB
	5V Speisung vorhanden ist. Die LED D1 und D2 signalisieren eine laufende Kommunikation zwischen
	Computer und Shock Detector Modul. Eine viert LED D3 signalisiert, dass ein Shock Event detektiert
	wurde. Zus�tzlich kann bei einem Shock Event ein Warnsignal an den Clcok Controller gesendet
	werden. Daf�r ist am Anschlussblock X2 ein Anschluss vorgesehen. Eine Supressordiode (Transil)
	sch�tzt den Mikrokontroller vor Spannungsimpulsen. Ein Taster dient der Konfiguration oder f�r
	Testzwecke und wurde als Reserve vorgesehen.
	
	
	\begin{figure}
		\centering
		%\vspace{-1cm}
		\includegraphics[width=23cm, angle=90]{img/HW/schema.png}
		\caption{Schema}
		\label{fig:schema}
	\end{figure}
	
	\paragraph{Layout}
	Das Layout besteht aus zwei Layern. Auf dem Top Layer werden die Bauteile platziert und die
	Leiterbahnen gef�hrt. Der Bottom Layer besteht gr�sstenteils aus einer Groundfl�che. Diese ist mit
	Vias zum Top Layer verbunden. Auskreuzungen auf dem Bottom Layer werden m�glichst vermieden und so
	kurz wie m�glich gehalten. Neben den Auskreuzungen befinden sich Beschriftungen der Anschl�sse und
	der wichtigsten Testpunkte auf dem Bottom Layer. Unter und neben dem Spannungsregler ist eine
	Kupferfl�che, so dass die W�rme abgeleitet werden kann. Die Transildiode wird m�glichst nahe am
	Anschlussblock X2 und die Leiterbahn direkt �ber das Pad der Transildiode gef�hrt (Application Note
	\cite{Transil}). Allgemein werden die Bauteile kompakt angeordnet und die Leiterbahnf�hrung kurz
	gehalten. Die St�tzkondensatoren werden m�glichst nahe an den ICs platziert.

	Top Layer, Bottom Layer und der Top Overlay (Best�ckung) sind in den Abbildungen \ref{fig:TopBot}
	bis \ref{fig:BotVal} abgebildet. Zus�tzlich wurde der Print in der 3D-Ansicht kontrolliert. Dazu
	konnten einige Step-Modelle von den Bauteileherstellern in Altium eingebunden werden. Von den
	restlichen Bauteilen wurde selber ein vereinfachtes 3D- Modell gezeichnet. Die \autoref{fig:3d}
	zeigt die 3D Ansicht des Prints. Die St�ckliste ist im Anhang \ref{chap:stueckliste} angef�gt. Die
	St�ckliste wurde eingeteilt in Bauteile, welche extern bestellt werden m�ssen, und solche, welche
	im Lagerbestand der Moser-Baer AG sind. Der Preis f�r extern zu bestellende Bauteile betr�gt
	ungef�hr Fr. 7.65 pro Shock Detector Modul.
		
	\begin{figure}[H]
		\begin{minipage}[hbt]{7cm}
			\centering
			\includegraphics[width=7cm,]{img/HW/top.png}
			\caption{Top Layer}
			\label{fig:TopBot}
		\end{minipage}
		\hfill
		\begin{minipage}[hbt]{7cm}
			\centering
			\includegraphics[width=7cm]{img/HW/bottom.png}
			\caption{Bottom Layer}
			\label{fig:TopBez}
		\end{minipage}
		\\[4ex]
		\begin{minipage}[hbt]{7cm}
			\centering
			\includegraphics[width=7cm]{img/HW/designator.png}
			\caption{Bezeichnungen Top Layer}
			\label{fig:BotBez}
		\end{minipage}
		\hfill
		\begin{minipage}[hbt]{7cm}
			\centering
			\includegraphics[width=7cm]{img/HW/comment.png}
			\caption{Werte Top Layer}
			\label{fig:BotVal}
		\end{minipage}
	\end{figure}
	
	\begin{figure}
		\centering
		%\vspace{-1cm}
		\includegraphics[width=13cm]{img/HW/3d_w.png}
		\caption{3D Ansicht des Printes}
		\label{fig:3d}
	\end{figure}


\part{Schlussteil}
\chapter{Zusammenfassung}
	Im Rahmen der vorliegenden Arbeit wurde die Machbarkeit einer Schock Detektor mittels Beschleunigungssensoren 
	f�r vernetzte Aussenuhren gekl�rt. 
	
	Die Schock Detektor f�r vernetzte Aussenuhren ist mittels Beschleunigungsmessung grunds�tzlich
	m�glich. Die Versuche mit piezoelektrischen Beschleunigungssensoren haben gezeigt, dass durch
	optimale Platzierung des Sensors am oder im Uhrwerk eine Unterscheidung zwischen Schock Event und
	Druck-Sogwelle eines Zuges realisierbar ist. Schl�ge auf die Scheibe haben grosse
	Beschleunigungen im dreistelligen Bereich am Uhrwerk zur
	Folge. Im Gegensatz dazu misst der an dieser Position befestigte Sensor bei Kr�ften, welche auf
	einer gr�sseren Fl�che wirken, nur kleine Beschleunigungen. Dies ist beispielsweise bei der Druck-Sogwelle eines Zuges
	der Fall. 
	
	In einem weiteren Schritt wurde ein Datenlogger mit dem MEMS Sensor H3LIS331DL entwickelt. Der 
	funktionierende Datenlogger hat gezeigt, dass die Realisierung eines Schock Detektors 
	g�nstig und einfach m�glich ist. Ein Algorithmus mit drei Parametern Schwellwert, 
	minimale Dauer und minimale vergangene Zeit seit dem letzten Schock wurde implementiert und
	erfolgreich an der Versuchsuhr getestet. 
	
	Zum Schluss wurden die verschiedenen M�glichkeiten, einen Schock Detektor im bestehenden 
	Clock Set zu integrieren, zusammengetragen. Die Vor- und Nachteile der L�sungen wurden diskutiert. 
	Die beste Realisierungsidee wurde umgesetzt. Das Schock Detektor Modul kann an die R�ckseite des Uhrwerks montiert werden
	und dient als Datenlogger mit Schnittstelle an einen Computer, kann aber auch an den Clock Controller angeschlossen 
	werden. Messungen an Bahnh�fen und �ber l�ngere Zeit sind unumg�nglich. Die Uhren haben 
	verschiedene Durchmesser und sind verschieden montiert. Mit dem Schock Detektor Modul 
	kann eine umfassendere Messreihe an verschiedenen Bahnh�fen und Aussenuhren durchgef�hrt werden. 
	
	Die folgende Anforderungsliste zeigt die zu Beginn der Arbeit definierten Anforderungen und deren Erf�llungsgrad 
	am Schluss der Arbeit. Die Anforderungen an das Produkt sind zwar Forderungen, mussten jedoch noch nicht
	zwingend im Rahmen dieser Arbeit erf�llt werden. So ist es nicht schlimm, so dass die Zeit nicht gereicht hat, 
	die Software so zu erweitern, dass der Alarm �ber SNMP abgesetzt werden kann. Abkl�rungen 
	zur Realisierung auf den bestehenden ATMEL- Kontrollern wurden insofern gemacht, dass die ungef�hre
	Auslastung der Betriebssysteme bekannt sind. Genauere Untersuchungen folgen, sobald die 
	Software auf dem Schock Detektor Modul in Betrieb ist. Zusammenfassend l�sst sich sagen, dass die 
	zu Beginn der Arbeit definierten Ziele erreicht wurden. Dar�ber hinaus wurde ein erster Prototyp 
	der Schock Detektor als "'Schock Detektor Modul"' realisiert. 
	\newpage
	
	\paragraph{Anforderungsliste}	\label{chap:anforderungen}
	\begin{longtable}{p{0.5cm} p{0.5cm} p{8.5cm} p{0.75cm} p{1.5cm}} \toprule
		\textbf{Nr.}	&  &\textbf{Anforderung} & \textbf{Prio} & \textbf{Erf�l-lungsgrad}\\	
		\midrule
		\endhead
		\multicolumn{3}{l}{\emph{Fortsetzung auf n�chster Seite}}	\\ \bottomrule \endfoot \endlastfoot
		\\
		&		&\multicolumn{2}{l}{\textbf{Anforderungen an die Machbarkeitsstudie}}							
		\\
		1 & 	&	Es ist zu �berpr�fen, ob mit einem Beschleunigungssensor 
		Vandalismus an einer Uhr (Besch�digung Scheibe, Uhrwerk) 
		erkannt werden kann												& F & \cellcolor{green}erf�llt \\ 
		2 &  	& 	Der Vandalismus muss von Windb�en und der Sog-Druckwelle 
		eines Zuges unterschieden werden k�nnen. 						& F & \cellcolor{green}erf�llt \\		
		3 & 	& 	Es sollen verschiedene Schock-Situationen getestet werden.		& F & \cellcolor{green}erf�llt \\
		& 3.1	&	Es soll ein geeigneter Messaufbau entwickelt werden, um 
		die Schock-Situationen zu simulieren. 							&F & \cellcolor{green}erf�llt \\
		& 3.2 & 	Der Messaufbau muss so gew�hlt werden, dass die Messungen 
		reproduzierbar sind. 											&F & \cellcolor{green}erf�llt \\
		4 & 	&	Falls die Detektion m�glich ist, soll ein geeigneter 
		Algorithmus entwickelt werden (z.B. in MATLAB). 				& F & \cellcolor{green}erf�llt \\
		5 & 	& 	Es soll abgekl�rt werden, ob die Implementierung auf dem
		vorhandenem ATMEL- Kontroller realisiert werden kann.			& F & \cellcolor{orange}teilweise erf�llt \\
		&	5.1	&	Falls die Implementierung auf dem ATMEL- Kontroller nicht
		realisiert werden kann, muss abgekl�rt werden, welche
		anderen Komponenten ben�tigt werden.							& F &\cellcolor{green}erf�llt \\
		6 &		& 	Der Stromverbrauch der Produkterweiterung soll abgekl�rt und
		dokumentiert werden.											& F &\cellcolor{green} erf�llt \\
		7 & 	& 	Es soll abgekl�rt werden, ob eine vorhandene Schnittstelle 
		(SPI, I2C, UART) verwendet werden kann.							& W &\cellcolor{green}erf�llt \\
		\\
		&		&\multicolumn{2}{l}{\textbf{Anforderungen an das Produkt}}								
		\\
		1 &		& 	Die Sensorik darf von aussen nicht sichtbar sein.	& F &\cellcolor{green}erf�llt \\
		2 & 	& 	Der Alarm soll mittels SNMP �ber die bestehende Uhrsteuerung
		abgesetzt werden k�nnen.										& F &\cellcolor{red}nicht erf�llt \\
		3 &		&	Die Kosten f�r den Sensor liegen im Bereich von 5 bis maximal 10 Fr.
		pro St�ck.														& F & \cellcolor{green}erf�llt \\
		4 & 	& 	Das Sensormodul soll einfach in der Uhr montiert werden k�nnen.
		Die Platzverh�ltnisse sind insofern beschr�nkt, dass die
		Beleuchtung der Uhr nicht gest�rt wird (siehe Anforderung
		Produkt Nr. 1)													& F & \cellcolor{green}erf�llt \\
		5 & 	&	Das Produkt muss bei einer Betriebstemperatur von -30 bis +70�C 
		arbeiten.  														& F & \cellcolor{green}erf�llt \\
		\bottomrule					 	
		\caption{Anforderungen an die Machbarkeitsstudie und an das Produkt} 
		\label{tab:Anforderungen}
	\end{longtable}
	
	\paragraph{Lessons Learned}
	%\chapter{Lessons Learned}
		Eine Herausforderung dieser Arbeit war, dass Messungen auf dem Perron nicht m�glich waren.
		Deshalb musst eine andere L�sung gefunden werden, abzusch�tzen, was f�r einen Einfluss
		Durck-Sogwellen eines Zuges auf die Beschleunigung in der Aussenuhr hat. Dies hat gezeigt,
		dass nicht nur der auf den ersten Blick offensichtliche Weg zum Ziel f�hrt, sondern mit Geduld ein
		anderer Ansatz gefunden werden kann.
		
		Bei Inbetriebnahme des MEMS Beschleunigungssensors lag die Schwierigkeit darin, dass das
		Datenblatt des �fteren Fragen unbeantwortet liess. So war es zu Beginn nicht klar, wie genau die
		Beschleunigungsdaten interpretiert werden m�ssen.
		
		Bei der Softwareentwicklung des Datenloggers wurde das Gelernte �ber das Betriebssystem
		FreeRTOS das erste Mal richtig angewendet. Die effizientere Inter-Task-Kommunikation mittels Task
		Notifications war jedoch ein neuer Ansatz. Es ist nie ausgelernt!




\begin{appendix}
\part{Anhang}
\includepdf[pages=1-2,addtotoc={1,chapter,0,Aufgabenstellung,chp:aufgabenstellung}]{pdf/aufgabenstellung.pdf}


%\chapter{Anforderungen}

Die folgende Liste zeigt die zu Beginn der Arbeit mit dem Industriepartner definierten
Anforderungen. Diese sind in Anforderungen an die Machbarkeitsstudie, und die Anforderungen an das
Produkt unterteilt. Die Anforderungen an das Produkt sind f�r die Machbarkeitsstudie noch nicht
relevant.
\\
\begin{longtable}{p{0.5cm} p{0.5cm} p{8.5cm} p{2cm} } \toprule
	\textbf{Nr.}	&  &\textbf{Anforderung} & \textbf{Priorit�t} \\	
	\midrule
	\endhead
	\multicolumn{3}{l}{\emph{Fortsetzung auf n�chster Seite}}	\\ \bottomrule \endfoot \endlastfoot
	\\
	  &		&\multicolumn{2}{l}{\textbf{Anforderungen an die Machbarkeitsstudie}}							
	\\
	1 & 	&	Es ist zu �berpr�fen, ob mit einem Beschleunigungssensor 
				Vandalismus an einer Uhr (Besch�digung Scheibe, Uhrwerk) 
				erkannt werden kann												& Forderung \\ 
	2 &  	& 	Der Vandalismus muss von Windb�en und der Sog-Druckwelle 
				eines Zuges unterschieden werden k�nnen. 						& Forderung \\		
	3 & 	& 	Es sollen verschiedene Schock-Situationen getestet werden.		& Forderung \\
	  & 3.1	&	Es soll ein geeigneter Messaufbau entwickelt werden, um 
				die Schock-Situationen zu simulieren. 							& Forderung \\
	  & 3.2 & 	Der Messaufbau muss so gew�hlt werden, dass die Messungen 
				reproduzierbar sind. 											& Forderung \\
	4 & 	&	Falls die Detektion m�glich ist, soll ein geeigneter 
				Algorithmus entwickelt werden (z.B. in MATLAB). 				& Forderung \\
	5 & 	& 	Es soll abgekl�rt werden, ob die Implementierung auf dem
				vorhandenem ATMEL- Kontroller realisiert werden kann.			& Forderung \\
	  &	5.1	&	Falls die Implementierung auf dem ATMEL- Kontroller nicht
				realisiert werden kann, muss abgekl�rt werden, welche
				anderen Komponenten ben�tigt werden.							& Forderung \\
	6 &		& 	Der Stromverbrauch der Produkterweiterung soll abgekl�rt und
				dokumentiert werden.											& Forderung \\
	7 & 	& 	Es soll abgekl�rt werden, ob eine vorhandene Schnittstelle 
				(SPI, I2C, UART) verwendet werden kann.							& Forderung \\
	\\
	 &		&\multicolumn{2}{l}{\textbf{Anforderungen an das Produkt}}								
	\\
	1 &		& 	Die Sensorik darf von aussen nicht sichtbar sein.				& Forderung \\
	2 & 	& 	Der Alarm soll mittels SNMP �ber die bestehende Uhrsteuerung
				abgesetzt werden k�nnen.										& Forderung \\
	3 &		&	Die Kosten f�r den Sensor liegen im Breich von 5 bis maximal 10 Fr.
				pro St�ck.														& Forderung \\
	4 & 	& 	Das Sensormodul soll einfach in der Uhr montiert werden k�nnen.
				Die Platzverh�ltnisse sind insofern beschr�nkt, dass die
				Beleuchtung der Uhr nicht gest�rt wird (siehe Anforderung
				Produkt Nr. 1)													& Forderung \\
	5 & 	&	Das Produkt muss bei einer Betriebstemperatur von -30 bis +70�C 
				arbeiten.  														& Forderung \\
	\bottomrule					 	
	\caption{Anforderungen an die Machbarkeitsstudie und an das Produkt} 
	\label{tab:Anforderungen}
\end{longtable}

\includepdf[page=1, angle=90, addtotoc={1,chapter,0,St�ckliste,chap:stueckliste}]{pdf/stueckliste.pdf}
%%-------------------------------------------------------------------------------
% $HeadURL: http://hb9etc.ch/svn/pluess/tex/da_doku/anhang_ztrafo.tex $
% $Revision: 861 $
% $Author: tobias $
% $Date: 2013-12-23 21:15:48 +0100 (Mon, 23 Dec 2013) $
%-------------------------------------------------------------------------------


\chapter{z-Transformation} \index{z-Transformation|(}
Auf den folgenden Seiten soll kurz und knapp die Theorie �ber die
z-Transformation aufgefrischt werden. Um das ganze m�glichst kompakt zu halten,
werden s�mtliche Beispiele und weitere Erl�uterungen zu den gemachten S�tzen
weggelassen, es wird aber dennoch versucht, die z-Transformation so vollst�ndig
wie m�glich zu erkl�ren. F�r weitergehende Informationen hierzu sei auf
\cite{Frey2008, Lunze2, Froriep} verwiesen.

Die z-Transformation entspricht der Laplace-Transformation, ist jedoch f�r
diskrete Signale vorgesehen.
Die bilaterale z-Transformation ist mit
\begin{equation}
  X(z) = \sum \limits_{k=-\infty}^{\infty} x[k] \, z^{-k}
\end{equation}
definiert. Meist -- auch in der vorliegenden Arbeit -- wird aber die
unilaterale z-Transformation benutzt; diese wird
auch rechtsseitige oder einseitige z-Transformation genannt. Sie ist mit
\begin{equation}
  X(z) = \sum \limits_{k=0}^{\infty} x[k] \, z^{-k}
\end{equation}
definiert. In beiden F�llen ist $z \in \mathbb{C}$.

Die Funktion $X(z)$ heisst Bildfunktion, $x[k]$ heisst
Originalfunktion. Die z-Transformation kann mit $\lap{ x[k] }{ X(z) }$
geschrieben werden; ebenso ist auch $X(z) = \ztransz{x[k]}$ m�glich.


\section{Konvergenzbereich} \index{z-Transformation!Konvergenzbereich}

Die z-Transformierte einer Funktion $x[k]$ existiert nicht zwangsweise f�r
jeden beliebigen Wert von $z$, sondern es gibt auch hier �hnlich wie bei der
Laplace-Transformation bestimmte Bereiche, wo die z-Transformierte existiert
und solche Bereiche, wo sie nicht existiert. Man spricht auch hier wieder vom
sogenannten Konvergenzbereich.

Zun�chst f�hrt man sich die formale Definition der (unilateralen)
z-Transformation
\[
X(z) = \sum \limits_{k=0}^{\infty} x[k] \, z^{-k}
\]
noch einmal vor Augen. Da $z$ eine beliebige Zahl ist, die auch komplex sein
kann, kann sie i.A. als
\begin{equation}
  z = a \, \E^{\j\,b}
\end{equation}
geschrieben werden mit $a,b \in \mathbb{R}$. In diesem Fall ist dann
$a=\abs{z}$ und $b=\arc{z}$. F�r $z^{-k}$ gilt:
\begin{equation}
  z^{-k} = a^{-k} \, \E^{-\j\,b\,k}
\end{equation}
Man erkennt sofort: wenn $a>1$ ist, dann wird f�r wachsende $k$ die
Potenz $a^{-k}$ immer kleiner, und zwar in exponentiellem Masse. Sofern $x[k]$
h�chstens eine exponentiell wachsende Funktion ist, muss die z-Transformierte
also auf jeden Fall existieren, und zwar f�r alle $z$, f�r die $\abs{z}>1$ gilt.
Man erkennt hieraus also: f�r rechtsseitige Funktionen ist der Konvergenzbereich
der z-Transformierten die gesamte komplexe Ebene ausserhalb des Einheitskreises.

F�r Funktionen, die zeitlich begrenzt sind, d.h. nur in einem Intervall
$[k_0, k_1]$ verschieden von 0 sind, und ausserhalb dieses Intervalls
verschwinden (wie z.B. eine Rechteckfunktion), existiert die z-Transformierte
sogar auf der gesamten komplexen Ebene, ausgenommen dem Ursprung. F�r eine
solche Funktion lautet die z-Transformierte n�mlich:
\begin{equation}
  X(z) = \sum \limits_{k=k_0}^{k_1} x[k] \, z^{-k} = x[k_0] \, z^{-k_0} +
  x[k_0 + 1] \, z^{-k_0 - 1} + x[k_0 + 2] \, z^{-k_0 - 2} + \ldots +
  x[k_1] \, z^{-k_1}
\end{equation}
Es gibt also endlich viele Glieder, und daher muss die z-Transformierte auf
jeden Fall konvergieren. Da die Exponenten von $z$ jedoch allesamt negativ sind,
geh�rt der Ursprung der komplexen Ebene aufgrund der Beziehung
\begin{equation}
  z^{-k} = \frac{1}{z^k}
\end{equation}
nicht zum Konvergenzbereich dazu aufgrund der Division durch 0.

Die Funktion $x[k]$ muss allerdings auch noch bestimmten Bedingungen gen�gen,
damit die z-Transformierte existiert. Und zwar darf $x[k]$ h�chstens von
exponentiellem Typ sein. Dies heisst: sei
\begin{equation}
  x[k] < a \, \E^{b\,k}
\end{equation}
und wenn nun f�r jeses $k$ ein $a$ und ein $b$ existiert, f�r die diese
Bedingung wahr ist, dann existiert auch die z-Transformierte. Also:
\begin{equation}
\exists ~ a,b \Leftrightarrow x[k] < a \, \E^{b\,k} ~ \forall k
\end{equation}
Etwas einfacher
kann man diese
Regel auch so formulieren: wenn $x[k]$ nicht schneller als eine
Exponentialfunktion w�chst, dann wird auch eine z-Transformierte existieren.
Funktionen wie $x[k]=k!$ oder $x[k]=\E^{k^2}$ sind daher nicht
z-Transformierbar, da sie schneller als jede Exponentialfunktion wachsen.

\italictitle{Anmerkung}
F�r linksseitige Funktionen $x[k]$ die nur f�r $k<0$ verschieden von 0 sind,
kann eine �hnliche �berlegung angestellt werden wie f�r die rein rechtsseitigen
Funktionen. F�r eine solche linksseitige Funktion gilt n�mlich f�r die
z-Transformierte:
\[
X(z) = \sum \limits_{k=-\infty}^{0} x[k] \, z^{-k}
\]
die Exponenten von $z$ sind somit allesamt positiv. Eine solche unendliche
Reihe kann aber nur dann �berhaupt konvergieren, wenn $\abs{z}<1$ ist. Daher ist
der Konvergenzbereich einer solchen linksseitigen Funktion das innere des
Einheitskreises.


\section{Eigenschaften der z-Transformation}
\index{z-Transformation!Eigenschaften}

\subsection{Linearit�t} \index{z-Transformation!Linearit�t}
Die z-Transformation ist eine lineare Transformation. D.h., es ist:
\begin{equation}
  \ztransz{ a\,x[k] + b\,y[k] } = \ztransz{ a\,x[k] } + \ztransz{ b\,y[k] }
\end{equation}
Zudem gilt auch:
\begin{equation}
  \ztransz{ a\,x[k] } = a \cdot \ztransz{ x[k] }
\end{equation}

\subsection{Verschiebungssatz} \index{z-Transformation!Verschiebungssatz}
Bei einer Verschiebung der Originalfunktion um $k_0$ Zeiteinheiten nach rechts
muss die Bildfunktion entsprechend mit $z^{-k_0}$ multipliziert werden:
\begin{equation}
  \ztransz{ x[k - k_0] } = z^{-k_0} \cdot \ztransz{ x[k] }
\end{equation}
Dies gilt jedoch nur, wenn $x[k] = 0$ f�r $k < 0$ ist. Wenn dies nicht zutrifft,
dann gilt folgendes:
\begin{equation}
  \ztransz{ x[k - k_0] } = z^{-k_0} \cdot \left( X(z) + \sum \limits_{i=1}^{k_0} x[-i] \, z^i \right)
\end{equation}
Der Grund hierf�r ist, dass bei einer Verschiebung nach rechts auf der linken
Seite noch weitere Funktionswerte hinzukommen. Da allerdings i.d.R. nur die
unilaterale z-Transformation benutzt wird und damit $x[k] = 0$ f�r $k < 0$ ist,
ist diese Regel eher von untergeordneter Bedeutung.

Bei einer Verschiebung nach links gilt allerdings immer:
\begin{alignat}{2}
  \ztransz{ x[k] } &= X(z) \\
  \ztransz{ x[k + 1] } &= z \cdot \left( X(z) - x[0] \right) \\
  \ztransz{ x[k + 2] } &= z^2 \cdot \left( X(z) - x[0] - x[1] \, z^{-1} \right) \\
  \ztransz{ x[k + 3] } &= z^3 \cdot \left( X(z) - x[0] - x[1] \, z^{-2} - x[2] \, z^{-3} \right)
\end{alignat}
oder allgemein:
\begin{equation}
  \ztransz{ x[k + k_0] } = z^{k_0} \cdot \left( X(z) - \sum \limits_{i=0}^{k - 1} x[k] \, z^{-i} \right)
\end{equation}

\subsection{D�mpfungssatz} \index{z-Transformation!D�mpfung}
Bei einer D�mpfung eines Signals im Zeitbereich muss folgende Regel beachtet
werden: sei \lap{$X(z)}{x[k]}$, dann gilt
\begin{equation}
  \ztransz{ a^{-k} \, x[k] } = X(a \, z)
\end{equation}
was direkt durch Einsetzen in die Formel f�r die z-Transformation folgt:
\[
  \ztransz{ a^{-k} \, x[k] } =
  \sum \limits_{k=0}^{\infty} x[k] \, a^{-k} \, z^{-k} =
  \sum \limits_{k=0}^{\infty} x[k] \cdot \left(a \, z \right)^{-k}
\]
Dies hat allerdings einen Einfluss auf den Konvergenzbereich. Denn damit die
Summe konvergieren kann, muss $\abs{a\,z} > 1$ sein. Daher ist der
Konvergenzbereich der z-Transformierten: $z>\abs{\frac{1}{a}}$.

\subsection{Lineare Gewichtung} \index{z-Transformation!lineare Gewichtung}
Bei einer linear mit der Zeit gewichteten Funktion gilt folgender Satz:
\begin{equation}
  \ztransz{ k \, x[k] } = -z \cdot \frac{d}{dz} \, X(z)
\end{equation}
Denn die z-Transformierte einer solch gewichteten Funktion lautet
\begin{alignat*}{2}
  \ztransz{ k \, x[k] } &= \sum \limits_{k=0}^{\infty} k\,x[k]\,z^{-k} &&\\
  &= 0 + x[1]\,z^{-1} + 2\,x[2]\,z^{-2} + 3\,x[3]\,z^{-3} + \ldots &&
\end{alignat*}
w�hrend f�r die nicht gewichtete Funktion
\begin{alignat*}{2}
  X(z) &= \sum \limits_{k=0}^{\infty} x[k]\,z^{-k} &&\\
  &= x[0] + x[1]\,z^{-1} + x[2]\,z^{-2} + x[3]\,z^{-3} + \ldots && \quad\vline \frac{d}{dz} \\
  &= 0 - x[1]\,z^{-2} - 2\,x[2]\,z^{-3} - 3\,x[3]\,z^{-4} + \ldots && \quad\vline \cdot (-z) \\
  &= 0 + x[1]\,x^{-1} + 2\,x[2]\,z^{-2} + 3\,x[3]\,z^{-3} + \ldots &&
\end{alignat*}
gilt. Daraus folgt:
\[
 \underbrace{ 0 + x[1]\,z^{-1} + 2\,x[2]\,z^{-2} + 3\,x[3]\,z^{-3} + \ldots }_{ \ztransz{k\,x[k]}} =
 -z \cdot \frac{d}{dz} \, \ztransz{ x[k] }
\]

\subsection{Zeitinversion}\index{z-Transformation!Zeitinversion}
F�r negative $k$ gilt:
\begin{equation}
  \ztransz{ x[-k] } = X\left(\frac{1}{z}\right)
\end{equation}
Denn es ist
\[
  \ztransz{ x[-k] } = \sum \limits_{i=-\infty}^{0} x[k] \, z^{k}
\]
und da
\[
  z^k = \frac{1}{z^{-k}}
\]
ist, folgt hieraus der Zeitinversionssatz.

\subsection{Diskrete Ableitung}\index{z-Transformation!diskrete Ableitung}
Die diskrete Ableitung ist mit
\begin{equation}
  \frac{d}{dz} \, x[k] = x[k] - x[k-1]
\end{equation}
definiert. F�r die z-Transformierte davon gilt:
\begin{equation}
  \ztransz{ x[k] - x[k-1] } = X(z) \cdot \frac{z-1}{z}
\end{equation}
Dies geht aus der folgenden �berlegung mit Hilfe des Zeitverschiebungssatzes
hervor.
\begin{alignat*}{2}
  \ztransz{ x[k] - x[k - 1]} &= \sum \limits_{k=0}^{\infty} \left( x[k] - x[k-1] \right) \, z^{-k} &&\\
  &= \sum \limits_{k=0}^{\infty} x[k]\, z^{-k} - \sum \limits_{k=0}^{\infty} x[k-1]\, z^{-k} &&\\
  &= X(z) \cdot X(z) \, z^{-1} &&\\
  &= X(z) \cdot \left( 1 - z^{-1} \right) &&\\
  &= X(z) \cdot \left( 1 - \frac{1}{z} \right) &&\\
  &= X(z) \cdot \frac{z-1}{z} &&
\end{alignat*}

\subsection{Diskrete Integration}\index{z-Transformation!diskrete Integration}
Das diskrete Integral $I[k]$ einer Funktion $x[k]$ ist wie folgt definiert:
\begin{equation}
  I[k] = \sum \limits_{i=0}^k x[i]
\end{equation}
Die z-Transformierte davon lautet wie folgt:
\begin{equation}
  \ztransz{ \sum \limits_{i=0}^k x[i] } = X(z) \cdot \frac{z}{z-1}
\end{equation}
Dies kann damit begr�ndet werden, dass die Integration als die Faltung
\begin{equation}
  I = x[k] \ast 1
\end{equation}
dargestellt werden kann\footnote{Man beachte: das neutrale Element der Faltung
ist nicht 1, sondern $\delta[k]$.}. Unter Anwendung des Faltungssatzes erh�lt man so:
\begin{equation}
  \ztransz{ x[k] \ast 1 } = \underbrace{X(z) \vphantom{\frac{z}{z}} }_{\ztransz{x[k]}}
  \cdot \underbrace{ \frac{z}{z-1} }_{\ztransz{1}}
\end{equation}


\subsection{Faltungssatz} \index{z-Transformation!Faltung}
Bei einer diskreten Faltung zweier Signale im Zeitbereich werden die
Bildfunktionen miteinander Multipliziert:
\begin{equation}
  \ztransz{ x[k] * y[k] } = \ztransz{ x[k] } \cdot \ztransz{ y[k] }
\end{equation}


\subsection{Anfangswertsatz} \index{z-Transformation!Anfangswertsatz}
Ist die z-Transformierte $x(z)$ einer Funktion $x[k]$ bekannt, so kann direkt
ohne vorg�ngige R�cktransformation der Anfangswert $x[0]$ mit dem
Anfangswertsatz berechnet werden:
\begin{equation}
  \lim \limits_{k \rightarrow 0} x[k] = \lim \limits_{z \rightarrow \infty} X(z)
\end{equation}
Dies folgt unmittelbar durch Einsetzen in die Formel f�r die z-Transformation:
\[
  \lim \limits_{z\rightarrow\infty} X(z) =
  \lim \limits_{z\rightarrow\infty} \sum \limits_{k=0}^{\infty} x[k] \, z^{-k} =
  x[0]
\]
wenn $z$ gegen unendlich geht, dann nimmt n�mlich die Folge $z^{-k}$ so rasant
ab, dass in der Summe nur noch das 1. Glied $x[0]$ �brig bleibt und alle anderen
Glieder zu 0 werden.

\subsection{Endwertsatz} \index{z-Transformation!Endwertsatz}
Analog zum Anfangswertsatz gibt es auch eine M�glichkeit, den Endwert der
Funktion $x[k]$ ohne vorherige R�cktransformation zu berechnen. Dazu dient die
folgende Beziehung:
\begin{equation}
  \lim \limits_{k \rightarrow \infty} x[k] =
  \lim \limits_{z \rightarrow 1} \left(z - 1\right) \cdot X(z)
\end{equation}
Ist  der Endwert der Sprungantwort gesucht, dann kann dieser wie folgt berechnet
werden:
\begin{equation}
  \lim \limits_{z \rightarrow 1} \cancel{ \left(z - 1\right) } \cdot \frac{z}{\cancel{ z - 1} } \cdot X(z)
  = \lim \limits_{z \rightarrow 1} X(z)
\end{equation}
Dies liegt darin begr�ndet dass die Heaviside-Funktion ja die z-Transformierte
$\frac{z}{z-1}$ hat. Die Sprungantwort erh�lt man somit durch Multiplikation mit
$\frac{z}{z-1}$.


\section{Inverse z-Transformation} \index{z-Transformation!inverse}

\subsection{Geschlossene L�sung}

Die inverse z-Transformation kann theoretisch mit
\begin{equation}
  x[k] = \frac{1}{2\,\pi\,\j} \oint \limits_{(C)} X(z) \, z^{k-1} \, dz
\end{equation}
berechnet werden. Dabei ist die Kurve $C$ eine beliebige geschlossene Kurve in
der komplexen Ebene, die komplett innerhalb des Konvergenzbereichs der
z-Transformierten liegen muss und die den Ursprung sowie alle Pole der 
z-Transformierten $X(z)$ umschliesst.
Dies ist jedoch sehr aufwendig. Man behilft sich daher mit
�hnlichen Verfahren wie bei der Laplace-Transformation, d.h. die Bildfunktion
wird mittels Partialbruchzerlegung in bekannte Bildfunktionen zerlegt, die dann
einfach zur�cktransformiert werden k�nnen.

Dabei ist es oft von Vorteil, nicht direkt die Bildfunktion $X(z)$ zur�ck zu
transformieren, sondern die Funktion $\tilde{X}(z) = \frac{X(z)}{z}$. Man
zerlegt also $\tilde{X}(z)$ in Partialbr�che. Anschliessend werden die
einzelnen Partialbr�che wieder mit $z$ multipliziert und dann erst kann die
R�cktransformation der Partialbr�che erfolgen.

Der Grund f�r diesen Umweg ist die im Vergleich zur Laplace-Transformation
prinzipiell andere Form der z-Transformierten. I.d.R. ist es nicht sehr einfach,
eine z-Transformierte direkt r�ckzutransformieren, mit diesem Umweg wird diese
Angelegenheit aber etwas vereinfacht.


\subsection{Rekursive R�cktransformation} \index{z-Transformation!inverse, rekursive}
Nicht in allen F�llen gelingt eine exakte R�cktransformation einer
z-Transformierten, weil u.U. die Partialbruchzerlegung nicht m�glich oder zu
aufwendig ist, oder aber man ist nur an den numerischen Werte interessiert,
nicht aber an der geschlossenen Form der R�cktransformierten.
In einem solchen Fall kann auf die rekursive R�cktransformation
zur�ckgegriffen werden. Wohl liefert sie die einzelnen diskreten Werte der
R�cktransformierten, aber man erh�lt keinen geschlossenen Term zur Beschreibung
des Resultats.

F�r die rekursive R�cktransformation einer Funktion $X(z)$ muss die
z-Transformierte zun�chst in eine Form wie die folgende gebracht werden:
\begin{equation}
  X(z) = \frac{ \sum \limits_{i=0}^M b_i \, z^i }{ \sum \limits_{i=0}^N a_i \, z^i }
  = \frac{ b_0 + b_1\,z + b_2\,z^2 + b_3\,z^3 + \ldots + b_M\,z^M}{a_0 + a_1\,z + a_2\,z^2 + a_3\,z^3 + \ldots + a_N\,z^N}
  \label{zform}
\end{equation}
Die Exponenten von $z$ m�ssen also allesamt positiv sein. Ist dieser Schritt
vollzogen, dann kann die rekursive R�cktransformation mit folgender Vorschrift
vorgenommen werden:
\begin{equation}
  x[k] = \begin{cases}
  0 & k < 0 \\
  \frac{1}{a_R} \cdot \left( b_{R-k} - \sum \limits_{i=1}^R a_{R-i} \, x[k - i] \right) & k \geq 0
  \end{cases}
\end{equation}
Hierbei ist $R$ die Anzahl der Koeffizienten. Ist der Z�hlergrad $M$ verschieden
vom Nennergrad $N$, dann gilt:
\begin{equation}
  R = \max \left\{ M, N \right\} \label{nwert}
\end{equation}
Mit dieser rekursiven R�cktransformation ist es nun also m�glich, die
numerischen Werte der einzelnen St�tzstellen einer Funktion im Zeitbereich zu
berechnen. Voraussetzung hierf�r ist dann allerdings, dass man die
vorhergehenden St�tzstellen bereits berechnet hat. Werte $x[k]$ f�r $k < 0$
werden einfach zu 0 angenommen, damit ist dies eine unilaterale
R�cktransformation.


\section{G�ngige Korrespondenzen der z-Transformation}
\begin{table}
  \centering
  \begin{tabular}{l l l l l} \toprule
  Zeitbereich                     & Bildbereich                   && Zeitbereich                      & Bildbereich \\ \midrule
  $\displaystyle 1$               & $\displaystyle \frac{z}{z-1}$ && $\displaystyle k$                & $\displaystyle \frac{z}{\left(z-1\right)^2}$ \\[1.1em]
  $\displaystyle a^k$             & $\displaystyle \frac{z}{z-a}$ && $\displaystyle \E^{\alpha\,k}$   & $\displaystyle \frac{z}{z-\E^{\alpha}}$ \\[1.1em]
  $\displaystyle a^{k-1}$         & $\displaystyle \frac{1}{z-a}$ && $\displaystyle \cos \omega\,k$  & $\displaystyle \frac{z^2 - z\cos\omega}{z^2 -2\,z\cos\omega + 1}$ \\[1.1em]
  $\displaystyle \delta[k]$       & $1$                           && $\displaystyle \sin\omega\,k$   & $\displaystyle \frac{z\sin\omega}{z^2 - 2\,z\cos\omega + 1}$ \\ \bottomrule
  \end{tabular}
  \caption{G�ngige Korrespondenzen}
\end{table}

\index{z-Transformation|)}

%%-------------------------------------------------------------------------------
% $HeadURL: http://hb9etc.ch/svn/pluess/tex/da_doku/anhang_glossar.tex $
% $Revision: 861 $
% $Author: tobias $
% $Date: 2013-12-23 21:15:48 +0100 (Mon, 23 Dec 2013) $
%-------------------------------------------------------------------------------

\chapter{Glossar} \index{Glossar} \index{Abk�rzungsverzeichnis}
%\markboth{Glossar}{Glossar}
%\addcontentsline{toc}{chapter}{Glossar}
\label{glossar}

\begin{description}[style=multiline,leftmargin=2.5cm]
  \item[\bf Queue] Eine Liste mit Eintr�gen, welche abgearbeitet werden sollen. 
  Eine Queue kann einen Aufrufer blockieren, falls sie voll oder leer ist. \index{Queue} \index{Queue|see{Queue}}
  
 
\end{description}
                 % anhang d, glossar (?)
\end{appendix}

{
\raggedright
\setlength{\bibsep}{3mm}
\interlinepenalty=10000
\bibliography{bibliografie/bibliografie}
\interlinepenalty=100
}

\newpage
\clearpage

\listoftables                            % tabellenverzeichnis
\addcontentsline{toc}{chapter}{\listtablename}
\clearpage

\listoffigures                           % bilderverzeichnis
\addcontentsline{toc}{chapter}{\listfigurename}
\clearpage

%\lstlistoflistings
%\addcontentsline{toc}{chapter}{Listing-Verzeichnis}
%\clearpage



\linespread{0.99}
\printindex                              % sachverzeichnis

\end{document}

%-------------------------------------------------------------------------------
